\chapter{Abstract}

\section{Abstract}
Breast cancer continues to be a major global health challenge, with estrogen
signaling playing a pivotal role in its onset and progression.
Recently, \glspl{crna} have emerged as important regulators of gene expression
and signaling pathways, holding significant potential in understanding breast
cancer biology.
Due to their stability and tissue-specific expression, \glspl{crna} are also
being explored as promising biomarkers for cancer diagnosis and therapy.

In this thesis, I employed the \gls{nf-circrna} pipeline to analyze two
datasets centered on estrogen signaling and breast cancer in mice.
The first dataset explored the impact of \gls{esr1} and \gls{cyp19}
overexpression on cancer development throughout reproductive senescence.
The second dataset examined the effects of common breast cancer treatments,
\gls{let} and \gls{tam}, on cancer progression during the same phase.

Through the analysis, I found that the detected \glspl{crna} positions
exhibited slight variations across different tools and samples.
To address this, I introduced a novel parameter, \textit{max shift}, to enhance
the robustness of \gls{crna} detection across datasets.

Additionally, the differential expression analysis revealed several
\glspl{crna} significantly associated with \gls{esr1} expression and the
treatment methods.
Of these, three \glspl{crna} demonstrated a particularly strong correlation
with \gls{tam} and \gls{let} treatment.

Furthermore, known \gls{mirna}-gene interactions were leveraged to identify the
target genes of these differentially expressed \glspl{crna}.
Gene set enrichment analysis revealed that the target genes are involved in key
biological processes and pathways linked to the effects of \gls{let} and
\gls{tam} treatment.

While these findings are promising, further validation of the \textit{max
    shift} parameter and the identified candidate \glspl{crna} is essential.
Future research should prioritize systematically assessing the impact of the
\textit{max shift} parameter on \gls{crna} detection outcomes and
experimentally validating the candidate \glspl{crna} with additional datasets
and methods.

\section{Kurzzusammenfassung}

Deutsche Version des Abstracts.
