\chapter{Abstract}

\section{Abstract}
Breast cancer continues to be a major global health challenge, with estrogen
signaling playing a pivotal role in its onset and progression.
Recently, \glspl{crna} have emerged as important regulators of gene expression
and signaling pathways, holding significant potential in understanding breast
cancer biology.
Due to their stability and tissue-specific expression, \glspl{crna} are also
being explored as promising biomarkers for cancer diagnosis and therapy.

In this thesis, I employed the \gls{nf-circrna} pipeline to analyze two
datasets centered on estrogen signaling and breast cancer in mice.
The first dataset explored the impact of \gls{esr1} and \gls{cyp19}
overexpression on cancer development throughout reproductive senescence.
The second dataset examined the effects of common breast cancer treatments,
\gls{let} and \gls{tam}, on cancer progression during the same phase.

Through the analysis, I found that the detected \glspl{crna} positions
exhibited slight variations across different tools and samples.
To address this, I introduced a novel parameter, \textit{max shift}, to enhance
the robustness of \gls{crna} detection across datasets.

Additionally, the differential expression analysis revealed several
\glspl{crna} significantly associated with \gls{esr1} expression and the
treatment methods.
Of these, three \glspl{crna} demonstrated a particularly strong correlation
with \gls{tam} and \gls{let} treatment.

Furthermore, known \gls{mirna}-gene interactions were leveraged to identify the
target genes of these differentially expressed \glspl{crna}.
Gene set enrichment analysis revealed that the target genes are involved in key
biological processes and pathways linked to the effects of \gls{let} and
\gls{tam} treatment.

While these findings are promising, further validation of the \textit{max
    shift} parameter and the identified candidate \glspl{crna} is essential.
Future research should prioritize systematically assessing the impact of the
\textit{max shift} parameter on \gls{crna} detection outcomes and
experimentally validating the candidate \glspl{crna} with additional datasets
and methods.

\newpage

\section{Kurzzusammenfassung}

Brustkrebs bleibt weltweit eine bedeutende gesundheitliche Herausforderung,
wobei die Östrogensignalübertragung eine zentrale Rolle in seiner Entstehung
und Progression spielt.
In den letzten Jahren haben sich \glspl{crna} als wichtige Regulatoren der
Genexpression und von Signalwegen herauskristallisiert, mit einem erheblichen
Potenzial für das Verständnis der Brustkrebsbiologie.
Aufgrund ihrer Stabilität und ihres gewebespezifischen Expressionsmusters
werden \glspl{crna} zudem als vielversprechende Biomarker für die Krebsdiagnose
und -therapie untersucht.

In dieser Arbeit habe ich die \gls{nf-circrna}-Pipeline verwendet, um zwei
Datensätze zu analysieren, die sich mit der Östrogensignalübertragung und
Brustkrebs bei Mäusen beschäftigen.
Der erste Datensatz untersuchte, wie die Überexpression von \gls{esr1} und
\gls{cyp19} die Krebsentwicklung während der reproduktiven Seneszenz
beeinflusst.
Der zweite Datensatz befasste sich mit den Auswirkungen der gängigen
Brustkrebsbehandlungen Letrozol und \Gls{tam} auf das Fortschreiten des
Brustkrebses während der reproduktiven Seneszenz.

Die Analyse zeigte, dass die Positionen der erkannten \glspl{crna} zwischen
verschiedenen Erkennungsmethodens und Proben leicht variieren.
Um dies zu beheben, habe ich einen neuen Parameter, \textit{max shift},
eingeführt, um die Robustheit der \gls{crna}-Erkennung zu erhöhen.

Darüber hinaus identifizierte die differentielle Expressionsanalyse mehrere
\glspl{crna}, die signifikant mit der \gls{esr1}-Expression und den
Behandlungsmethoden assoziiert waren.
Unter diesen \glspl{crna} stachen drei besonders hervor, da sie eine starke
Assoziation mit der \Gls{tam}- und Letrozol-Behandlung aufwiesen.

Zudem wurden bekannte \glsfmtfull{mirna}-Gen-Interaktionen genutzt, um Zielgene
der differentiell exprimierten \glspl{crna} zu identifizieren.
Die Gen-Set-Anreicherungsanalyse zeigte, dass diese Zielgene in zentrale
biologische Prozesse und Signalwege involviert sind, die mit den Auswirkungen
der Letrozol- und \Gls{tam}-Behandlung in Verbindung stehen.

Obwohl diese Ergebnisse vielversprechend sind, ist eine weitere Validierung des
\textit{max shift}-Parameters und der identifizierten Kandidaten-\glspl{crna}
erforderlich.
Zukünftige Studien sollten sich darauf konzentrieren, systematisch die
Auswirkungen des \textit{max shift}-Parameters auf die \gls{crna}-Erkennung zu
untersuchen und die Kandidaten-\glspl{crna} mit weiteren Datensätzen und
experimentellen Methoden zu validieren.
