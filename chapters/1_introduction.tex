\chapter{Introduction}

Breast cancer remains one of the most significant challenges in modern
healthcare, as the most frequently diagnosed cancer among women worldwide.
In 2022 alone, approximately 2.3 million new cases were recorded, along with
666,000 deaths, illustrating its substantial impact on global health.
Accounting for 23.8\% of all cancer diagnoses in
women\supercite{bray_global_2024,ferlay_global_2024}, breast cancer continues
to be the leading cause of cancer-related death for women.
Projections estimate an alarming rise in incidence, with cases expected to
reach 4.4 million annually by 2070, further straining healthcare
systems\supercite{lei_global_2021}.

Despite advancements in screening, early detection, and treatment, disparities
in access to care persist, particularly in low-resource settings.
These disparities contribute to late-stage diagnoses and higher mortality rates
in such regions\supercite{wilkinson_understanding_2022,ginsburg_breast_2020}.
Globally, the age-standardized mortality rate for breast cancer remains high at
approximately 12.65 deaths per 100,000 women per year, highlighting the ongoing
challenges in managing this complex
disease\supercite{bray_global_2024,ferlay_global_2024}.
Given these persistent challenges, significant efforts are needed in research,
public health strategies, and healthcare delivery to improve patient
outcomes\supercite{desantis_breast_2019}.

At the molecular level, a growing body of research focuses on circular RNAs
(circRNAs), a novel class of RNA molecules characterized by their covalently
closed loop structures.
Unlike linear RNAs, circRNAs exhibit increased stability and resistance to
degradation, making them attractive as biomarkers and potential therapeutic
targets in various diseases, including
cancer\supercite{ma_circular_2020,hoque_exploring_2023,wilusz_circular_2017}.
Initially thought to be byproducts of mRNA splicing, circRNAs are now
recognized for their regulatory roles in gene expression, cellular processes,
and disease mechanisms\supercite{cherubini_foxp1_2019,wilusz_360_2018}.

The unique properties of circRNAs — stability, abundance, and tissue-specific
expression — make them promising candidates for diagnostic and therapeutic
applications, especially in cancer.
CircRNAs have demonstrated potential as biomarkers for tumor progression and
treatment response\supercite{bao_prognostic_2020,ren_construction_2017}, with
their stability in bodily fluids enhancing their utility in non-invasive liquid
biopsies\supercite{bao_prognostic_2020,zhang_circular_2018}.
In breast cancer specifically, circRNAs have emerged as key players in disease
progression, particularly in relation to estrogen receptor (ER) signaling.
Dysregulated circRNA expression is observed in breast cancer, where circRNAs
function as competing endogenous RNAs (ceRNAs), sponging miRNAs involved in
critical pathways such as ER
signaling\supercite{nair_circular_2016,xu_circrna_2022}.
Certain circRNAs, such as circRNA-SFMBT2, have even been linked to tamoxifen
resistance, a significant hurdle in treating ER-positive breast
cancer\supercite{li_circrna-sfmbt2_2023}.
Moreover, circRNAs like circTADA2As and circFBXL5 have been shown to regulate
miRNAs involved in key signaling pathways that promote cell proliferation and
metastasis, highlighting their clinical
relevance\supercite{xu_circtada2as_2019,gao_hsa_circrna_0006528_2019}.

This thesis aims to deepen our understanding of the role of circRNAs in breast
cancer, with a particular focus on their involvement in estrogen signaling and
tumor progression.
To this end, sequencing data from two mouse model studies conducted by Furth et
al.
\supercite{furth_esr1_2023,furth_overexpression_2023}
are analyzed, exploring circRNA expression profiles in relation to variables
such as age, transgene induction, and anti-hormone treatments.
The nf-core/circrna pipeline\supercite{digby_nf-corecircrna_2023} is employed
to identify and quantify circRNAs from the RNA-seq data, followed by
differential expression analysis to identify circRNAs differentially expressed
across experimental conditions.
Additionally, this thesis compares different methods for circRNA
identification, quantification, and differential expression analysis, assessing
their performance and reliability in capturing circRNA dynamics.

\medskip
\noindent The thesis is structured to guide the reader through the key concepts,
methodologies, and findings related to the role of circRNAs in breast cancer.

The Background section (\cref{chap:background}) provides the foundational
knowledge necessary to understand the subsequent chapters on methodology and
analysis.
It begins with an overview of breast cancer epidemiology (\cref{sec:brca}),
types (\cref{sec:brca_types}), causes and risk factors
(\cref{sec:brca_risk-factors}), diagnosis (\cref{sec:brca_diagnosis}), and
treatment (\cref{sec:brca_treatment}), followed by an introduction to estrogen
signaling (\cref{sec:estrogen_signaling}), which is central to breast cancer
biology.
Next, the basic mechanisms of gene expression are explained
(\cref{sec:gene_expression}), laying the groundwork for understanding the
complex roles of RNA molecules.
This is followed by a dedicated overview of circular RNAs
(\cref{sec:circrnas}), covering their biogenesis
(\cref{sec:circrna_biogenesis}), types (\cref{sec:circrna_types}), functions
(\cref{sec:circrna_functions}), and potential applications
(\cref{sec:circrna_applications}) in both molecular biology and cancer
research.

In the Materials and Methods section (\cref{chap:materials_and_methods}), the
thesis details the mouse model studies by Furth et al.
\supercite{furth_esr1_2023,furth_overexpression_2023} (\cref{sec:data}),
which form the basis for the circRNA expression
analysis.
Additionally, this section describes the nf-core/circrna
pipeline\supercite{digby_nf-corecircrna_2023} used to process and analyze the
RNA-seq data (\cref{sec:nf-core_circrna}).

The Results and Discussion section then presents a comparison of different
tools used for circRNA identification, quantification, and differential
expression analysis.
This is followed by an in-depth analysis of circRNA expression profiles
observed in the mouse models, with a focus on the implications for breast
cancer progression and estrogen signaling.
