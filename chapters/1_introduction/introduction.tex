\chapter{Introduction}

Breast cancer remains one of the most significant challenges in modern
healthcare, as the most frequently diagnosed cancer among women worldwide.
In 2022 alone, approximately 2.3 million new cases were recorded, along with
666,000 deaths, illustrating its substantial impact on global health.
Accounting for 26.2\% of all cancer diagnoses in
women\supercite{bray_global_2024,ferlay_global_2024}, breast cancer continues
to be the leading cause of cancer-related death for women.
Projections estimate an alarming rise in incidence, with cases expected to
reach 4.4 million annually by 2070, further straining healthcare
systems\supercite{lei_global_2021}.

Despite advancements in screening, early detection, and treatment, disparities
in access to care persist, particularly in low-resource settings.
These disparities contribute to late-stage diagnoses and higher mortality rates
in such regions\supercite{wilkinson_understanding_2022,ginsburg_breast_2020}.
Globally, the age-standardized mortality rate for breast cancer remains high at
approximately 12.65 deaths per 100,000 women per year, highlighting the ongoing
challenges in managing this complex
disease\supercite{bray_global_2024,ferlay_global_2024}.
Given these persistent challenges, significant efforts are needed in research,
public health strategies, and healthcare delivery to improve patient
outcomes\supercite{desantis_breast_2019}.

At the molecular level, gene expression is a fundamental biological process
through which the genetic information encoded in \gls{dna} is utilized to produce
functional molecules, primarily proteins, that perform various cellular
activities\supercite{salmena_cerna_2011}.
This process occurs in two main stages: transcription, where \gls{dna} is transcribed
into \gls{mrna}, and translation, during which ribosomes translate \gls{mrna}
into proteins\supercite{salmena_cerna_2011,tay_multilayered_2014}.
Regulation of gene expression is tightly controlled at multiple levels,
including transcriptional, post-transcriptional, and translational stages,
ensuring that proteins are synthesized in appropriate amounts according to
cellular
needs\supercite{poliseno_coding-independent_2010,tay_multilayered_2014}.

In recent years, advancements in our understanding of post-transcriptional
regulation have led to the identification of \glspl{cerna}
\supercite{salmena_cerna_2011,tay_multilayered_2014}.
These molecules interact with \glspl{mirna} to modulate gene expression
\supercite{salmena_cerna_2011,li_long_2017}.
\glspl{mirna} are small, non-coding \glspl{rna} that bind to target
\glspl{mrna},
leading to their degradation or inhibition of
translation\supercite{salmena_cerna_2011,tay_multilayered_2014}.
\Glspl{cerna}, which include long non-coding \glspl{rna}, pseudogenes, and
\glspl{crna}, can
"sponge" \glspl{mirna} by binding to them, effectively preventing \glspl{mirna}
from silencing their target
\glspl{mrna}\supercite{salmena_cerna_2011,poliseno_coding-independent_2010}.
This mechanism increases the expression of \gls{mrna} targets, making
\glspl{cerna} crucial regulators in maintaining cellular homeostasis and
influencing disease processes, including
cancer\supercite{salmena_cerna_2011,vo_landscape_2019}.

\Glspl{crna} are unique among \gls{rna} species due to their covalently closed
loop
structure, which provides distinct properties such as enhanced stability and
resistance to exonucleases\supercite{vo_landscape_2019}.
These features position \glspl{crna} as promising candidates for diagnostic and
therapeutic applications across various
diseases\supercite{ma_circular_2020,hoque_exploring_2023,wilusz_circular_2017}.
They have shown potential as biomarkers for tumor progression and treatment
response\supercite{bao_prognostic_2020,ren_construction_2017}, with their
stability in bodily fluids enhancing their applicability in non-invasive liquid
biopsies\supercite{bao_prognostic_2020,zhang_circular_2018}.

In the context of breast cancer, \glspl{crna} have emerged as significant
contributors to disease progression, particularly in relation to \gls{er} signaling.
Dysregulated \gls{crna} expression has been observed in breast cancer, where
\glspl{crna} act as \glspl{cerna}, sponging \glspl{mirna} that are critical to
pathways such as \gls{er} signaling\supercite{nair_circular_2016,xu_circrna_2022}.
Notably, certain \glspl{crna}, such as \gls{crna}-SFMBT2, have been linked to
tamoxifen resistance, posing a substantial challenge in the treatment of
\gls{er+} breast cancer\supercite{li_circrna-sfmbt2_2023}.
Furthermore, \glspl{crna} like circTADA2As and circFBXL5 have been demonstrated
to regulate \glspl{mirna} involved in key signaling pathways that promote cell
proliferation and metastasis, underscoring their clinical
relevance\supercite{xu_circtada2as_2019,gao_hsa_circrna_0006528_2019}.

This thesis aims to deepen our understanding of the role of \glspl{crna} in
breast cancer, with a particular focus on their involvement in estrogen
signaling and tumor progression.
To this end, sequencing data from two mouse model studies conducted by Furth et
al.
\supercite{furth_esr1_2023,furth_overexpression_2023}
are analyzed, exploring \gls{crna} expression profiles in relation to variables
such as age, transgene induction, and anti-hormone treatments.
The \gls{nf-circrna} pipeline\supercite{digby_nf-corecircrna_2023} is employed
to identify and quantify \glspl{crna} from the \gls{rna-seq} data, followed by
differential expression analysis to identify \glspl{crna} differentially
expressed across experimental conditions.
Additionally, this thesis compares different methods for \gls{crna}
identification, quantification, and differential expression analysis, assessing
their performance and reliability in capturing \gls{crna} dynamics.

\medskip
\noindent The thesis is structured to guide the reader through the key
concepts,
methodologies, and findings related to the role of \glspl{crna} in breast
cancer.

The Background section (\cref{chap:background}) provides the foundational
knowledge necessary to understand the subsequent chapters on methodology and
analysis.
It begins with an overview of breast cancer epidemiology (\cref{sec:brca}),
types (\cref{sec:brca_types}), causes and risk factors
(\cref{sec:brca_risk-factors}), diagnosis (\cref{sec:brca_diagnosis}), and
treatment (\cref{sec:brca_treatment}), followed by an introduction to estrogen
signaling (\cref{sec:estrogen_signaling}), which is central to breast cancer
biology.
Next, the basic mechanisms of gene expression are explained
(\cref{sec:gene_expression}), laying the groundwork for understanding the
complex roles of \gls{rna} molecules.
This is followed by a dedicated overview of circular \glspl{rna}
(\cref{sec:circrnas}), covering their biogenesis
(\cref{sec:circrna_biogenesis}), types (\cref{sec:circrna_types}), functions
(\cref{sec:circrna_functions}), and potential applications
(\cref{sec:circrna_applications}) in both molecular biology and cancer
research.

In the Materials and Methods section (\cref{chap:materials_and_methods}), the
thesis details the mouse model studies by Furth et al.
\supercite{furth_esr1_2023,furth_overexpression_2023} (\cref{sec:data}),
which form the basis for the \gls{crna} expression
analysis.
Additionally, this section describes the \gls{nf-circrna}
pipeline\supercite{digby_nf-corecircrna_2023} which was used to process and
analyze the \gls{rna-seq} data (\cref{sec:nf-core_circrna}).

The Results and Discussion section then presents a comparison of different
tools used for \gls{crna} identification, quantification, and differential
expression analysis.
This is followed by an in-depth analysis of \gls{crna} expression profiles
observed in the mouse models, with a focus on the implications for breast
cancer progression and estrogen signaling.
