\chapter{Introduction}

Breast cancer remains one of the most pressing challenges in modern healthcare,
as the most frequently diagnosed cancer among women worldwide.
In 2022 alone, approximately 2.3 million new cases were recorded, alongside
666,000 deaths, underscoring its significant global health impact.
Accounting for 26.2\% of all cancer diagnoses in
women\supercite{bray_global_2024,ferlay_global_2024}, breast cancer continues
to be the leading cause of cancer-related mortality in women.
Projections indicate an alarming increase in incidence, with cases expected to
reach 4.4 million annually by 2070, placing further strain on healthcare
systems\supercite{lei_global_2021}.

Despite advancements in screening, early detection, and treatment, disparities
in access to care persist, particularly in low-resource settings.
These inequities contribute to late-stage diagnoses and higher mortality rates
in such regions\supercite{wilkinson_understanding_2022,ginsburg_breast_2020}.
Globally, the age-standardized mortality rate for breast cancer remains high,
at approximately 12.65 deaths per 100,000 women per year, highlighting the
ongoing difficulties in managing this complex
disease\supercite{bray_global_2024,ferlay_global_2024}.
Addressing these challenges requires continued research, improvements in public
health strategies, and advances in healthcare delivery to improve patient
outcomes\supercite{desantis_breast_2019}.

At the molecular level, gene expression is a fundamental biological process by
which the genetic information encoded in \gls{dna} is used to produce
functional molecules, primarily proteins, that drive cellular
activities\supercite{salmena_cerna_2011}.
This process involves two key stages: transcription, where \gls{dna} is
transcribed into \gls{mrna}, and translation, during which ribosomes synthesize
proteins from the \gls{mrna}
template\supercite{salmena_cerna_2011,tay_multilayered_2014}.
Gene expression is tightly regulated at multiple levels—transcriptional,
post-transcriptional, and translational—to ensure proteins are produced in
appropriate amounts in response to cellular
demands\supercite{poliseno_coding-independent_2010,tay_multilayered_2014}.

In recent years, research into post-transcriptional regulation has identified
\glspl{cerna} as key players in modulating gene
expression\supercite{salmena_cerna_2011,tay_multilayered_2014}.
These molecules interact with \glspl{mirna} to influence gene
regulation\supercite{salmena_cerna_2011,li_long_2017}.
\Glspl{mirna}, small
non-coding \glspl{rna}, bind to target \glspl{mrna}, leading to their
degradation or inhibiting their
translation\supercite{salmena_cerna_2011,tay_multilayered_2014}. \Glspl{cerna},
which include long non-coding \glspl{rna}, pseudogenes, and \glspl{crna}, act
as "sponges" for \glspl{mirna}, preventing them from silencing their target
\glspl{mrna}, thus increasing the expression of the targeted
\glspl{mrna}\supercite{salmena_cerna_2011,poliseno_coding-independent_2010}.
This regulatory mechanism positions \glspl{cerna} as crucial modulators of
cellular homeostasis and disease progression, including
cancer\supercite{salmena_cerna_2011,vo_landscape_2019}.

\Glspl{crna} stand out among \gls{rna} species due to their covalently closed
loop structure, conferring enhanced stability and resistance to
exonucleases\supercite{vo_landscape_2019}.
These properties make \glspl{crna} promising candidates for diagnostic and
therapeutic applications across a range of
diseases\supercite{ma_circular_2020,hoque_exploring_2023,wilusz_circular_2017}.
In particular, they have shown potential as biomarkers for tumor progression
and treatment response, with their stability in bodily fluids enhancing their
utility in non-invasive liquid
biopsies\supercite{bao_prognostic_2020,zhang_circular_2018}.

In the context of breast cancer, \glspl{crna} have emerged as important
contributors to disease progression, particularly through their role in
\gls{er} signaling.
Dysregulated \gls{crna} expression has been observed in breast cancer, where
\glspl{crna} act as \glspl{cerna}, sponging \glspl{mirna} that are critical to
pathways such as \gls{er}
signaling\supercite{nair_circular_2016,xu_circrna_2022}.
Notably, specific \glspl{crna}, such as \gls{crna}-SFMBT2, have been linked to
\gls{tam} resistance, presenting a major challenge in treating \gls{er+} breast
cancer\supercite{li_circrna-sfmbt2_2023}.
Additionally, \glspl{crna} like circTADA2As and circFBXL5 have been shown to
regulate \glspl{mirna} involved in key pathways that promote cell proliferation
and metastasis, further underscoring their clinical
relevance\supercite{xu_circtada2as_2019,gao_hsa_circrna_0006528_2019}.

This thesis seeks to deepen our understanding of the role of \glspl{crna} in
breast cancer, with a particular focus on their involvement in estrogen
signaling and tumor progression.
To this end, sequencing data from two mouse model studies conducted by
\textcite{furth_esr1_2023,furth_overexpression_2023} are analyzed to explore
\glspl{crna} expression profiles in relation to variables such as age,
transgene induction, and anti-hormone treatments.
The \gls{nf-circrna} pipeline\supercite{digby_nf-corecircrna_2023} is used to
identify and quantify \glspl{crna} from the \gls{rna-seq} data, followed by
differential expression analysis to identify \glspl{crna} differentially
expressed across experimental conditions.
Furthermore, this thesis compares different methods for \gls{crna}
identification, quantification, and differential expression analysis, assessing
their performance and reliability in capturing \gls{crna} dynamics.
Lastly, interactions between differentially expressed \glspl{crna} and
\glspl{mirna} are predicted to gain insights into potential regulatory
functions of the identified \glspl{crna}.

\medskip \noindent This thesis is structured to guide the reader through the
key concepts, methodologies, and findings related to the role of \glspl{crna}
in breast cancer.

The Background section (\cref{chap:background}) provides the foundational
knowledge necessary to understand the subsequent chapters on methodology and
analysis.
It begins with an overview of breast cancer epidemiology (\cref{sec:brca}),
types (\cref{sec:brca_types}), causes and risk factors
(\cref{sec:brca_risk-factors}), diagnosis (\cref{sec:brca_diagnosis}), and
treatment (\cref{sec:brca_treatment}), followed by an introduction to estrogen
signaling (\cref{sec:estrogen_signaling}), which is central to breast cancer
biology.
Next, the basic mechanisms of gene expression are explained
(\cref{sec:gene_expression}), laying the groundwork for understanding the
complex roles of \gls{rna} molecules.
This is followed by a dedicated overview of circular \glspl{rna}
(\cref{sec:circrnas}), covering their biogenesis
(\cref{sec:circrna_biogenesis}), types (\cref{sec:circrna_types}), functions
(\cref{sec:circrna_functions}), and potential applications
(\cref{sec:circrna_applications}) in both molecular biology and cancer
research.


In the Materials and Methods section (\cref{chap:materials_and_methods}), the
thesis details the mouse model studies by
\textcite{furth_esr1_2023,furth_overexpression_2023} (\cref{sec:data}), which
form the basis for the \gls{crna} expression analysis.
Additionally, this section describes the \gls{nf-circrna}
pipeline\supercite{digby_nf-corecircrna_2023} which was used to process and
analyze the \gls{rna-seq} data (\cref{sec:nf-core_circrna}).

The Results and Discussion section then presents a comparison of different
tools used for \gls{crna} identification, quantification, and differential
expression analysis.
This is followed by an in-depth analysis of \gls{crna} expression profiles
observed in the mouse models, with a focus on the implications for breast
cancer progression and estrogen signaling.
