\chapter{Abstract}

\section{Abstract}
Breast cancer continues to be a major global health challenge, with estrogen
signaling playing a pivotal role in its onset and progression.
Recently, \glspl{crna} have emerged as important regulators of gene expression
and signaling pathways, holding significant potential in understanding breast
cancer biology.
Due to their stability and tissue-specific expression, \glspl{crna} are also
being explored as promising biomarkers for cancer diagnosis and therapy.

In this thesis, I employed the \gls{nf-circrna} pipeline to analyze two
datasets centered on estrogen signaling and breast cancer in mice.
The first dataset explored the impact of \gls{esr1} and \gls{cyp19}
overexpression on cancer development throughout reproductive senescence.
The second dataset examined the effects of common breast cancer treatments,
\gls{let} and \gls{tam}, on cancer progression during the same phase.

Through the analysis, I found that the detected \gls{crna} positions exhibited
slight variations across different tools and samples.
To address this, I introduced a novel parameter, \textit{max shift}, which
substantially enhances the robustness of \gls{crna} detection across datasets.
This parameter could potentially improve the reliability of results in future
\gls{crna} studies, providing more consistent outputs from the same input data.

Additionally, the differential expression analysis revealed several
\glspl{crna} significantly associated with \gls{esr1} expression and the
treatment methods.
Notably, three \glspl{crna} demonstrated a particularly strong correlation with
\gls{tam} and \gls{let} treatments.
These \glspl{crna} may serve as potential biomarkers for monitoring treatment
success or offer insights into the mechanism of action of these therapies.

Furthermore, known \gls{mirna}-gene interactions were leveraged to identify the
target genes of these differentially expressed \glspl{crna}.
Gene set enrichment analysis revealed that the target genes are involved in key
biological processes and pathways linked to the effects of \gls{let} and
\gls{tam} treatment.

While these findings are promising, further validation of the \textit{max
    shift} parameter and the identified candidate \glspl{crna} is essential.
Future research should prioritize systematically assessing the impact of the
\textit{max shift} parameter on \gls{crna} detection outcomes and
experimentally validating the candidate \glspl{crna} with additional datasets
and methods.

\newpage

\section{Kurzzusammenfassung}

Brustkrebs bleibt weltweit eine bedeutende gesundheitliche Herausforderung,
wobei die \\Östrogensignalgebung eine zentrale Rolle in seiner Entstehung und
Progression spielt.
In den letzten Jahren haben sich \Glsfmtlongpl{crna} (zirkuläre \Glspl{rna})
als wichtige Regulatoren der Genexpression und Signalwege herauskristallisiert,
mit einem erheblichen Potenzial für das Verständnis der Brustkrebsbiologie.
Aufgrund ihrer Stabilität und ihres gewebespezifischen Expressionsmusters
werden \Glsfmtlongpl{crna} zudem als vielversprechende Biomarker für die
Krebsdiagnose und -therapie untersucht.

In dieser Arbeit habe ich die \gls{nf-circrna}-Pipeline verwendet, um zwei
Datensätze zu analysieren, die sich mit der Östrogensignalgebung und Brustkrebs
bei Mäusen beschäftigen.
Der erste Datensatz untersuchte, wie die Überexpression von \Gls{esr1} und
\Gls{cyp19} die Krebsentwicklung während der reproduktiven Seneszenz
beeinflusst.
Der zweite Datensatz befasste sich mit den Auswirkungen der gängigen
Brustkrebsbehandlungen Letrozol und \Gls{tam} auf das Fortschreiten des
Brustkrebses in derselben Phase.

Die Analyse zeigte, dass die erkannten circRNA-Positionen zwischen
verschiedenen Tools und Proben leicht variierten.
Um dies zu beheben, habe ich einen neuen Parameter, max shift, eingeführt, der
die Robustheit der circRNA-Erkennung erheblich verbessert.
Dieser Parameter könnte zukünftig die Zuverlässigkeit von Ergebnissen in
circRNA-Studien steigern, da mit denselben Eingabedaten robustere Resultate
erzielt werden können.

Darüber hinaus identifizierte die differentielle Expressionsanalyse mehrere
\Glsfmtlongpl{crna}, die substantiell mit der \gls{esr1}-Expression und den
angewendeten Behandlungsmethoden assoziiert waren.
Besonders drei \Glsfmtlongpl{crna} wiesen eine starke Korrelation mit der
\Gls{tam}- und Letrozol-Behandlung auf.
Diese \Glsfmtlongpl{crna} könnten als potenzielle Marker für den
Behandlungserfolg dienen oder helfen, den Wirkungsmechanismus der Therapien
besser zu verstehen.

Obwohl diese Ergebnisse vielversprechend sind, ist eine weitere Validierung des
max shift-Parameters und der identifizierten Kandidaten-\Glsfmtlongpl{crna}
notwendig.
Zukünftige Studien sollten sich darauf konzentrieren, die Auswirkungen des max
shift-Parameters systematisch zu untersuchen und die
Kandidaten-\Glsfmtlongpl{crna} mit weiteren Datensätzen und Methoden
experimentell zu validieren.
