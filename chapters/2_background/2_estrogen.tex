\section{Estrogen signaling in breast cancer}
\label{sec:estrogen_signaling}

Estrogen signaling plays a pivotal role in the development and progression of
breast cancer, particularly in \gls{er+} subtypes.
The primary mechanism of action involves the binding of estrogen to \glspl{er}
(\gls{era} and \gls{erb}), which are transcription factors that regulate gene
expression associated with cell proliferation, differentiation, and survival
\supercite{misawa_estrogen-related_2015,lattouf_lkb1_2016}.
The activation of these receptors can lead to the transcription of genes that
promote tumor growth and metastasis, thereby establishing a direct link between
estrogen signaling and breast cancer pathogenesis
\supercite{feng_cross-talk_2020}.

In \gls{er+} breast cancer, estrogen signaling is often enhanced through
various intracellular signaling pathways, including the \gls{pi3k}/AKT and
\gls{mapk} pathways.
These pathways can be activated by growth factors such as \gls{her2}, which is
frequently overexpressed in aggressive breast
cancers\supercite{bratton_regulation_2010,salmeron-hernandez_bcas2_2019}.
The crosstalk between estrogen signaling and these pathways not only amplifies
the proliferative effects of estrogen but also contributes to the survival of
cancer cells under therapeutic stress, such as
chemotherapy\supercite{bratton_regulation_2010,george_hypoxia_2012}.
For instance, the interaction between \gls{era} and the \gls{pi3k}/AKT pathway
has been shown to suppress apoptosis and promote cell survival, which is
critical for tumor
progression\supercite{bratton_regulation_2010,george_hypoxia_2012}.

Moreover, the role of alternative estrogen receptors, such as \gls{era}36, has
emerged as a significant factor in breast cancer biology.
\gls{era}36 is a variant of the classical \gls{era} that mediates
rapid, non-genomic signaling responses to estrogen, influencing cell
proliferation and survival in both \gls{er+} and \gls{er-} breast cancer
cells\supercite{deng_er-36-mediated_2014,zhang_positive_2011}.
This receptor's activation can lead to enhanced sensitivity to estrogen, even
in cells that typically do not express classical estrogen receptors, thus
complicating treatment strategies\supercite{zhang_positive_2011}.

The influence of estrogen on breast cancer is further complicated by the
presence of co-regulators and other signaling molecules.
For example, the protein CAND1 has been implicated in the regulation of
estrogen signaling pathways, suggesting that it may play a role in metastasis
in \gls{er+} breast cancer\supercite{alhammad_bioinformatics_2022}.
Additionally, the interaction of estrogen with various growth factor signaling
pathways can lead to the upregulation of genes associated with cancer stem cell
properties, thereby promoting tumor heterogeneity and resistance to
therapies\supercite{fillmore_estrogen_2010,xue_sox9fxyd3src_2019}.

\subsection{Important genes}

The following genes are essential components of the estrogen signaling pathway.
While there are many other genes involved in this pathway, these two genes have
been genetically modified in the datasets used in this thesis.
Further details on the genetic modifications can be found in
\cref{sec:datasets}.

\paragraph{\Glsfmtfull{cyp19}}
The \gls{cyp19} gene encodes the aromatase enzyme, which is essential for the
biosynthesis of estrogens from androgens (such as testosterone), playing a
pivotal role in estrogen signaling and breast cancer development.
Elevated \gls{cyp19} expression has been linked to poor survival outcomes in
\gls{er+} breast cancer patients, with high levels correlating with increased
metastasis and recurrence
rates\supercite{barros-oliveira_cyp19a1_2020,friesenhengst_elevated_2018}.
Genetic polymorphisms in \gls{cyp19}, such as the rs936306 C/T variant, have
been associated with variations in circulating estradiol levels, influencing
breast cancer risk\supercite{ghosh_potential_2012}.
Furthermore, \gls{cyp19} expression is regulated by various factors, including
\gls{fsh}, which enhances its transcription in granulosa cells, thereby
promoting estrogen
synthesis\supercite{savolainen-peltonen_estrogen_2018,li_microrna-7a2_2022}.
Dysregulation of \gls{cyp19} can lead to aberrant estrogen levels, contributing
to tumorigenesis and resistance to endocrine
therapies\supercite{dabydeen_comparison_2015}.
Thus, \gls{cyp19} serves as both a critical enzyme in estrogen biosynthesis and
a potential therapeutic target in breast cancer management.

\paragraph{\Glsfmtfull{esr1}}
The \gls{esr1} gene encodes the \gls{era} protein, which - as mentioned above -
is a critical regulator of estrogen signaling in breast cancer.
\Gls{esr1} mutations and amplifications are associated with resistance to
endocrine
therapies, such as tamoxifen, complicating treatment
outcomes\supercite{aguilar_biological_2010,jeselsohn_emergence_2014}.
Notably, \gls{esr1} mutations can lead to constitutively active receptors that
drive tumor growth independent of estrogen\supercite{toy_activating_2017}.
Additionally, \gls{esr1}'s expression is modulated by various factors,
including \glspl{mirna} and transcription factors, which can influence the
cancer phenotype and therapeutic responses\supercite{mansoori_mir-142-3p_2019}.
The gene's regulatory mechanisms, including chromatin remodeling and enhancer
activity, further illustrate its complex role in breast cancer
biology\supercite{powers_proteasome_2013,tomita_cluster_2015}.
Thus, \gls{esr1} serves as both a biomarker for prognosis and a target for
therapeutic intervention in \gls{er+} breast cancer.
