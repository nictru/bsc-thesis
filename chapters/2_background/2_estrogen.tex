\section{Estrogen signaling}

Estrogen signaling plays a pivotal role in the development and progression of
breast cancer, particularly in estrogen receptor-positive (ER+) subtypes.
The primary mechanism of action involves the binding of estrogen to estrogen
receptors (ER\textalpha{} and ER\textbeta{}), which are transcription factors
that regulate gene expression associated with cell proliferation,
differentiation, and survival
\supercite{misawa_estrogen-related_2015,lattouf_lkb1_2016}.
The activation of these receptors can lead to the transcription of genes that
promote tumor growth and metastasis, thereby establishing a direct link between
estrogen signaling and breast cancer pathogenesis
\supercite{feng_cross-talk_2020}.

In ER+ breast cancer, estrogen signaling is often enhanced through various
intracellular signaling pathways, including the phosphoinositide 3-kinase
(PI3K)/AKT and mitogen-activated protein kinase (MAPK) pathways.
These pathways can be activated by growth factors such as HER2, which is
frequently overexpressed in aggressive breast
cancers\supercite{bratton_regulation_2010,salmeron-hernandez_bcas2_2019}.
The crosstalk between estrogen signaling and these pathways not only amplifies
the proliferative effects of estrogen but also contributes to the survival of
cancer cells under therapeutic stress, such as
chemotherapy\supercite{bratton_regulation_2010,george_hypoxia_2012}.
For instance, the interaction between ER\textalpha{} and the PI3K/AKT pathway
has been shown to suppress apoptosis and promote cell survival, which is
critical for tumor
progression\supercite{bratton_regulation_2010,george_hypoxia_2012}.

Moreover, the role of alternative estrogen receptors, such as
ER-\textalpha{}36, has emerged as a significant factor in breast cancer
biology.
ER-\textalpha{}36 is a variant of the classical ER\textalpha{} that mediates
rapid, non-genomic signaling responses to estrogen, influencing cell
proliferation and survival in both ER+ and ER-negative breast cancer
cells\supercite{deng_er-36-mediated_2014,zhang_positive_2011}.
This receptor's activation can lead to enhanced sensitivity to estrogen, even
in cells that typically do not express classical estrogen receptors, thus
complicating treatment strategies\supercite{zhang_positive_2011}.

The influence of estrogen on breast cancer is further complicated by the
presence of co-regulators and other signaling molecules.
For example, the protein CAND1 has been implicated in the regulation of
estrogen signaling pathways, suggesting that it may play a role in metastasis
in ER+ breast cancer\supercite{alhammad_bioinformatics_2022}.
Additionally, the interaction of estrogen with various growth factor signaling
pathways can lead to the upregulation of genes associated with cancer stem cell
properties, thereby promoting tumor heterogeneity and resistance to
therapies\supercite{fillmore_estrogen_2010,xue_sox9fxyd3src_2019}.
