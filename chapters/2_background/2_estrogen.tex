\section{Estrogen signaling in breast cancer}
\label{sec:estrogen_signaling}

Estrogen signaling plays a pivotal role in the development and progression of
breast cancer, particularly in \gls{er+} subtypes.
The primary mechanism of action involves the binding of estrogen to \glspl{er}
(\gls{era} and \gls{erb}), which are transcription factors that regulate gene
expression associated with cell proliferation, differentiation, and survival
\supercite{misawa_estrogen-related_2015,lattouf_lkb1_2016}.
The activation of these receptors can lead to the transcription of genes that
promote tumor growth and metastasis, thereby establishing a direct link between
estrogen signaling and breast cancer pathogenesis
\supercite{feng_cross-talk_2020}.

In \gls{er+} breast cancer, estrogen signaling is often enhanced through
various intracellular signaling pathways, including the \gls{pi3k}/AKT and
\gls{mapk} pathways.
These pathways can be activated by growth factors such as \gls{her2}, which is
frequently overexpressed in aggressive breast
cancers\supercite{bratton_regulation_2010,salmeron-hernandez_bcas2_2019}.
The crosstalk between estrogen signaling and these pathways not only amplifies
the proliferative effects of estrogen but also contributes to the survival of
cancer cells under therapeutic stress, such as
chemotherapy\supercite{bratton_regulation_2010,george_hypoxia_2012}.
For instance, the interaction between \gls{era} and the \gls{pi3k}/AKT pathway
has been shown to suppress apoptosis and promote cell survival, which is
critical for tumor
progression\supercite{bratton_regulation_2010,george_hypoxia_2012}.

Moreover, the role of alternative estrogen receptors, such as \gls{era}36, has
emerged as a significant factor in breast cancer biology.
\gls{era}36 is a variant of the classical \gls{era} that mediates
rapid, non-genomic signaling responses to estrogen, influencing cell
proliferation and survival in both \gls{er+} and \gls{er-} breast cancer
cells\supercite{deng_er-36-mediated_2014,zhang_positive_2011}.
This receptor's activation can lead to enhanced sensitivity to estrogen, even
in cells that typically do not express classical estrogen receptors, thus
complicating treatment strategies\supercite{zhang_positive_2011}.

The influence of estrogen on breast cancer is further complicated by the
presence of co-regulators and other signaling molecules.
For example, the protein CAND1 has been implicated in the regulation of
estrogen signaling pathways, suggesting that it may play a role in metastasis
in \gls{er+} breast cancer\supercite{alhammad_bioinformatics_2022}.
Additionally, the interaction of estrogen with various growth factor signaling
pathways can lead to the upregulation of genes associated with cancer stem cell
properties, thereby promoting tumor heterogeneity and resistance to
therapies\supercite{fillmore_estrogen_2010,xue_sox9fxyd3src_2019}.

\subsection{Important genes}
\label{sec:important_genes}

The following genes are essential components of the estrogen signaling pathway.
While there are many other genes involved in this pathway, these two genes have
been genetically modified in the datasets used in this thesis.
Further details on the genetic modifications can be found in
\cref{sec:datasets}.

\subsubsection{\Glsfmtlong{cyp19}}
The \gls{cyp19} gene encodes the aromatase enzyme, which is essential for the
biosynthesis of estrogens from androgens (such as testosterone), playing a
pivotal role in estrogen signaling and breast cancer development.
Elevated \gls{cyp19} expression has been linked to poor survival outcomes in
\gls{er+} breast cancer patients, with high levels correlating with increased
metastasis and recurrence
rates\supercite{barros-oliveira_cyp19a1_2020,friesenhengst_elevated_2018}.
Genetic polymorphisms in \gls{cyp19}, such as the rs936306 C/T variant, have
been associated with variations in circulating estradiol levels, influencing
breast cancer risk\supercite{ghosh_potential_2012}.
Furthermore, \gls{cyp19} expression is regulated by various factors, including
\gls{fsh}, which enhances its transcription in granulosa cells, thereby
promoting estrogen
synthesis\supercite{savolainen-peltonen_estrogen_2018,li_microrna-7a2_2022}.
Dysregulation of \gls{cyp19} can lead to aberrant estrogen levels, contributing
to tumorigenesis and resistance to endocrine
therapies\supercite{dabydeen_comparison_2015}.
Thus, \gls{cyp19} serves as both a critical enzyme in estrogen biosynthesis and
a potential therapeutic target in breast cancer management.

\subsubsection{\Glsfmtfull{esr1}}
\label{sec:esr1}
The \gls{esr1} gene encodes the \gls{era} protein, which - as mentioned above -
is a critical regulator of estrogen signaling in breast cancer.
\Gls{esr1} mutations and amplifications are associated with resistance to
endocrine
therapies, such as \gls{tam}, complicating treatment
outcomes\supercite{aguilar_biological_2010,jeselsohn_emergence_2014}.
Notably, \gls{esr1} mutations can lead to constitutively active receptors that
drive tumor growth independent of estrogen\supercite{toy_activating_2017}.
Additionally, \gls{esr1}'s expression is modulated by various factors,
including \glspl{mirna} and transcription factors, which can influence the
cancer phenotype and therapeutic responses\supercite{mansoori_mir-142-3p_2019}.
The gene's regulatory mechanisms, including chromatin remodeling and enhancer
activity, further illustrate its complex role in breast cancer
biology\supercite{powers_proteasome_2013,tomita_cluster_2015}.
Thus, \gls{esr1} serves as both a biomarker for prognosis and a target for
therapeutic intervention in \gls{er+} breast cancer.

\subsection{Important treatment agents}
\label{sec:important_treatments}

The following treatment agents are commonly used in the management of \gls{er+}
breast cancer.
They are of special relevance in this study, as they have been administered to
some cohorts of the mouse models analyzed in the used datasets.
Details on the treatment regimens can be found in \cref{sec:datasets}.

\subsubsection{\Glsfmtlong{tam}}

Tamoxifen is a \gls{serm} that plays a crucial role in the management of
estrogen \gls{er+} breast cancer.
By competing with estrogen for binding to the \gls{er}, \gls{tam} inhibits the
proliferation of breast cancer cells that rely on estrogen signaling for
growth\supercite{radhi_tamoxifen_2023}.
This mechanism is particularly significant as approximately 70\% of breast
cancers express \gls{er}, making \gls{tam} a cornerstone of endocrine therapy
for these patients\supercite{wang_induced_2019}.
Additionally, \gls{tam} has been shown to induce apoptosis in \gls{er+} cancer
cells, further contributing to its anticancer
effects\supercite{li_tamoxifen_2017}.

Despite its efficacy, \gls{tam} resistance remains a significant challenge,
affecting nearly 40\% of patients\supercite{tuttle_novel_2012}.
Resistance mechanisms are multifaceted, involving alterations in estrogen
signaling pathways, cell cycle regulation, and growth factor receptor
pathways\supercite{zhuang_p21-activated_2015,mills_mechanisms_2018}.
Furthermore, \gls{tam}'s effects can vary depending on the tissue context,
exhibiting antiestrogenic properties in breast tissue while acting as an
estrogen agonist in other tissues\supercite{jovanovska_effects_2021}.
This duality underscores the complexity of \gls{tam}'s role in breast cancer
treatment and the need for ongoing research to overcome resistance and optimize
therapeutic strategies.

\subsubsection{\Glsfmtlong{let}}

\Gls{let} is a third-generation non-steroidal aromatase inhibitor that plays a
significant role in estrogen signaling and the treatment of estrogen \gls{er+}
breast cancer.
By inhibiting the aromatase enzyme, \gls{let} effectively reduces estrogen
production, leading to decreased estrogen levels in the body, which is crucial
for the growth of \gls{er+} tumors\supercite{priyadarsini_quality_2022}.
This mechanism of action is particularly beneficial for postmenopausal women,
as it targets the primary source of estrogen in this population—adrenal and
peripheral conversion of androgens to
estrogens\supercite{priyadarsini_quality_2022}.

Clinical studies have demonstrated that \gls{let} is more effective than
\gls{tam} in certain settings, particularly in the adjuvant treatment of
early-stage breast cancer, where it has been associated with improved
disease-free survival rates\supercite{jerusalem_continuous_2021}.
However, some tumors may develop resistance to \gls{let}, often linked to the
expression of estrogen-regulated genes and the tumor's hormonal environment
\supercite{lee_suppressed_2021}.
Furthermore, \gls{let}'s efficacy can be influenced by the presence of other
hormonal receptors, such as progesterone receptors, which may modulate the
tumor's response to treatment \supercite{lee_suppressed_2021}.
Overall, \gls{let} represents a critical advancement in endocrine therapy for
breast cancer, particularly for patients with \gls{er+} tumors.
