\section{Gene Expression}
\label{sec:gene_expression}

Gene expression is a complex process consisting of multiple steps that
culminate in the production of functional proteins.
This process is tightly regulated at various levels to ensure the accurate and
timely expression of genes in response to internal and external signals.
Here, we provide an overview of the key steps involved in gene expression,
focusing on the transcription and processing of pre-mRNA into mature mRNA.

\subsection{Transcription and Pre-mRNA Formation}
Gene expression begins with transcription, where RNA polymerase II synthesizes
a primary transcript of RNA from the DNA template.
This primary transcript, known as pre-mRNA, contains both exons (coding
sequences) and introns (non-coding sequences)\supercite{lee_mechanisms_2015}.
The pre-mRNA undergoes several modifications before it can be translated into a
protein.

\subsection{Capping and Polyadenylation}
One of the first modifications to occur during pre-mRNA processing is the
addition of a 5' cap, which consists of a modified guanine nucleotide.
This cap is crucial for mRNA stability, nuclear export, and initiation of
translation\supercite{topisirovic_cap_2011}.
Following transcription, the pre-mRNA is also polyadenylated at its 3' end,
where a stretch of adenine nucleotides is added.
This poly(A) tail enhances mRNA stability and facilitates its export from the
nucleus to the cytoplasm\supercite{passmore_roles_2022}.

\subsection{Splicing}
The splicing of pre-mRNA is a critical step that involves the removal of
introns and the joining of exons to form a continuous coding sequence.
This process is catalyzed by the spliceosome, a large and dynamic
ribonucleoprotein complex composed of small nuclear RNAs (snRNAs) and numerous
protein factors\supercite{lee_mechanisms_2015}.
The spliceosome recognizes specific sequences at the exon-intron boundaries,
facilitating the precise excision of introns and ligation of
exons\supercite{wang_splicing_2008}.

\subsubsection{Alternative Splicing}
A significant aspect of pre-mRNA processing is alternative splicing, which
allows a single gene to produce multiple mRNA isoforms by including or
excluding specific exons.
This process is regulated by various splicing factors that interact with
cis-acting elements in the
pre-mRNA\supercite{le_alternative_2015,murphy_therapeutic_2022}.
Approximately 95\% of human genes with multiple exons undergo alternative
splicing, contributing to the diversity of the proteome and enabling cells to
adapt to different physiological conditions\supercite{le_alternative_2015}.
The regulation of alternative splicing is influenced by the abundance and
activity of splicing factors, which can change in response to cellular
signals\supercite{wang_mechanism_2015}.

\subsection{Nuclear Export}
Once splicing is complete, the mature mRNA, now devoid of introns and
containing a 5' cap and a poly(A) tail, is exported from the nucleus to the
cytoplasm.
This export is facilitated by the nuclear cap-binding complex and other
RNA-binding proteins that ensure the mRNA is properly processed and ready for
translation\supercite{soucek_evolutionarily_2016}.
