\section{State of the art}
\Glspl{crna} have emerged as significant players in the
molecular landscape of breast cancer, particularly in relation to estrogen
signaling and tumor progression.
They have been implicated in various cellular processes, including the
regulation of gene expression and the modulation of signaling pathways
associated with cancer
development\supercite{li_circrna-sfmbt2_2023,tran_new_2020}.
The dysregulation of \glspl{crna} has been observed in breast cancer, where
they can act as \glspl{cerna}, sponging \glspl{mirna} and thereby influencing
the expression of target \glspl{mrna} involved in critical pathways such as
\gls{er} signaling\supercite{nair_circular_2016,xu_circrna_2022}.

\subsection{\Glsfmtshortpl{crna} in breast cancer}
Recent studies have highlighted specific \glspl{crna} that are associated with
breast cancer subtypes and their clinical implications.
For instance, \gls{crna}-SFMBT2 has been shown to orchestrate
\gls{er}\textalpha{} activation, contributing to \gls{tam} resistance in breast
cancer cells, thereby underscoring the role of \glspl{crna} in therapeutic
resistance\supercite{li_circrna-sfmbt2_2023}.
Additionally, \glspl{crna} such as circTADA2As and circFBXL5 have been
identified as regulators of \glspl{mirna} that target key signaling pathways,
including the SOCS3 and MAPK/ERK pathways, which are crucial for cell
proliferation and metastasis in breast
cancer\supercite{xu_circtada2as_2019,gao_hsa_circrna_0006528_2019}.
This suggests that \glspl{crna} not only participate in the pathogenesis of
breast cancer but also serve as potential biomarkers for diagnosis and
prognosis\supercite{liu_influence_2021,chen_circepsti1_2018}.

\subsection{\Glsfmtshortpl{crna} in estrogen signaling}
The interaction between \glspl{crna} and estrogen signaling is particularly
noteworthy.
Research has demonstrated that \gls{er+} breast cancer subtypes exhibit
specific \gls{crna} expression profiles that overlap with genes involved in
estrogen signaling pathways\supercite{nair_circular_2016}.
For example, \gls{crna}-000911 has been shown to suppress breast cancer cell
proliferation and invasion by sponging miR-449a, which in turn activates the
Notch1 signaling pathway, a known mediator of estrogen
signaling\supercite{wang_comprehensive_2018}.
Furthermore, the \gls{crna}-\gls{mirna}-\gls{mrna} regulatory networks
constructed in various studies have revealed that \glspl{crna} can modulate the
expression of genes related to estrogen response, oxidative stress, and
epigenetic modifications, thereby influencing tumor behavior and patient
outcomes\supercite{xu_circrna_2022,nair_circular_2016}.
