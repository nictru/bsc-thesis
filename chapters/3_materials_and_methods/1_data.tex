\section{Data}
\label{sec:data}

\subsection{Mouse models}
Mouse models allow the study of disease pathophysiology within the context of
physiological endocrine and immunologic function.

\subsection{Samples}
The data used in this study were published in two studies by P.
Furth et al.
in 2023
\supercite{furth_esr1_2023,furth_overexpression_2023}.

\subsubsection{Study design}
In both studies, the mice were genetically modified to overexpress Esr1 and
CYP19A1 in their mammary epithelial cells.
This genetic manipulation was designed to model increased estrogen signaling,
which is known to be a critical factor in breast cancer development, especially
after menopause.
The studies focused on inducing this overexpression at or around reproductive
senescence, which is a model for menopause in women, to better understand the
increased cancer risks in aging mammary tissue.

In both cases, doxycycline was used to induce the transgenes, simulating
elevated estrogen receptor and aromatase levels in older, post-reproductive
mice.
The mice were maintained on standard laboratory chow until around 12 months of
age, corresponding to middle age in humans, when transgene induction was
initiated.

\subsubsection{Differences in treatment}
The key difference between the two studies lies in the timing and duration of
the transgene induction, as well as the subsequent treatment with anti-hormonal
therapies.
In the anti-hormonal study, tamoxifen and letrozole were administered to mice
at different stages of their life (18-20 months), focusing on how the different
anti-hormonal treatments influenced mammary gland morphology and cancer
development in the context of high Esr1 and CYP19A1 levels.

In contrast, the aging study extended the treatment timeline further.
Mice were followed until they reached 30 months of age, and the study
specifically aimed to see if aging past reproductive senescence, without the
early administration of anti-hormonal agents, increased the likelihood of
ER-positive mammary adenocarcinomas.

\subsubsection{Sequencing process}
Both studies used RNA sequencing to analyze gene expression in the mammary
glands.
After euthanasia, thoracic mammary glands were flash frozen, and RNA was
extracted using a Direct Zol RNA miniprep kit.
The sequencing libraries were prepared from ribosome-depleted RNA and sequenced
using the Illumina NextSeq 550 platform with a single-end 75-bp read length.
Quality control was performed using FastQC, and reads were aligned to the mouse
reference genome (mm10) using STAR, with normalization and differential
expression analysis performed using tools like RUVSeq and DESeq2.

\subsubsection{Findings}

\paragraph{Anti-hormonal study}
The study found that Esr1 overexpression significantly increased the expression
of genes related to cell proliferation, particularly those linked to poor
prognostic indicators in early-stage hormone receptor-positive breast cancer.
Tamoxifen and letrozole treatments were able to reduce this elevated
proliferative signature, but only Esr1 mice showed substantial responsiveness
to tamoxifen during reproductive senescence.
Both models were responsive to letrozole before and after reproductive
senescence.

\paragraph{Aging study}
This study demonstrated that Esr1 overexpression in aged mice led
to a higher prevalence of estrogen receptor-positive mammary adenocarcinomas.
The tumors in Esr1-overexpressing mice were predominantly ER-positive, whereas
CYP19A1 mice exhibited a mix of ER-positive and adenosquamous carcinomas.
The Esr1 mice developed a persistent proliferative signature similar to that
found in human ER-positive breast cancer, supporting the role of aging and
prolonged estrogen exposure in the generation of these cancers.
