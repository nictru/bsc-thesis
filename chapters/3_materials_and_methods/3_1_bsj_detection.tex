\subsection{\gls{crna} detection}
\label{subsec:circrna_detection}
From a molecular perspective, \glspl{crna} are similar to their linear
counterparts, with the primary distinction being their circular structure.
As explained in \cref{sec:circrna_biogenesis}, \glspl{crna} are derived from a
linear precursor (identical to linear \gls{rna}) that undergoes back-splicing
to form a loop.
Thus, the only way to computationally identify \glspl{crna} is by detecting
reads spanning the \gls{bsj}, which cannot be accounted for by canonical
forward splicing.
\gls{nf-circrna} provides a total of seven tools for \gls{bsj}
detection.
However, only five of them are compatible with single-end sequencing data.
I will briefly discuss these five tools below.

\subsubsection{\Glsfmtlong{findcirc} (2013)}
\Gls{findcirc} is one of the pioneering tools for \gls{crna} detection,
specifically designed for identifying \glspl{crna} by leveraging \gls{rna-seq}
data.
It employs a novel alignment strategy, splitting reads that do not map linearly
to the genome into smaller fragments, which are then re-aligned to detect
\glspl{bsj}\supercite{memczak_circular_2013}.
This tool does not rely on any known annotations and processes \gls{rna-seq}
reads using Bowtie, an efficient aligner for identifying \gls{bsj} reads.
One of its key features is that it filters for \gls{bsj} reads, removing those
that align entirely to the genome, which makes it stand out from methods
relying heavily on gene annotations\supercite{memczak_circular_2013}.

While it has been effective in many studies, it may not capture \glspl{crna}
with non-canonical splice signals, which can limit its detection
capabilities\supercite{sekar_circular_2018,liu_prkra_2022}.
Nonetheless, \gls{findcirc} remains a popular choice for researchers due to its
ease of use and integration with \gls{rna-seq} workflows.

\subsubsection{\Glsfmtlong{segemehl} (2014)}
\Gls{segemehl} integrates a sensitive and flexible read-matching algorithm
based on
enhanced suffix arrays (ESAs).
It identifies \gls{bsj} reads through split-read mapping, where the
\gls{rna-seq} data is aligned using a dynamic programming strategy that
identifies splicing, trans-splicing, and fusion transcripts.
Its major advantage over other tools is its ability to map reads containing
multiple split sites, which boosts sensitivity for detecting \glspl{crna} even
in complex cases like long reads or sequences with
errors\supercite{hoffmann_multi-split_2014}.

While \gls{segemehl} has shown promise in identifying \glspl{crna}, its
performance can vary depending on the specific dataset and experimental
conditions\supercite{gao_ciri_2015,zeng_comprehensive_2017}.

\subsubsection{\Glsfmtlong{dcc} (2016)}
\Gls{dcc} relies on STAR as its underlying aligner, which maps \gls{rna-seq}
reads to
the
genome.
It uses a combination of filters to detect \glspl{bsj} by distinguishing
back-splice reads from linear splicing.
\Gls{dcc} integrates across replicates to minimize false positives and improve
the
precision of \gls{crna} detection.
What sets \gls{dcc} apart is its post-mapping step, where it not only detects
\glspl{crna} but also estimates their expression relative to the host gene
using read counts from junction and non-junction reads.
This makes it highly useful for comparing \gls{crna} expression across
conditions\supercite{cheng_specific_2016}.
\Gls{dcc} has been validated in numerous studies, demonstrating high
reliability and
accuracy in \gls{crna} detection\supercite{paraboschi_interpreting_2018}.

\subsubsection{\Glsfmtlong{cex2} (2016)}
\Gls{cex2} is another widely used tool that focuses on the detection of
\glspl{crna} by analyzing \gls{rna-seq} data.
It employs a unique strategy that combines both \gls{bsj} reads and linear
\gls{rna} reads to improve \gls{crna} identification.
\Gls{cex2} has demonstrated robust performance in various studies, often
ranking high in comparative evaluations against other \gls{crna} detection
tools\supercite{zeng_comprehensive_2017,nicolet_circular_2018}.
Its ability to provide detailed annotations and quantifications of \glspl{crna}
makes it a valuable resource for researchers\supercite{hansen_comparison_2016}.
%TODO: Update

\subsubsection{\Glsfmtlong{ciri2} (2018)}
\Gls{ciri2} improves upon the original \gls{ciri}\supercite{gao_ciri_2015} by
using a \gls{mle} based on multiple seed matching to detect \gls{bsj} reads.
This algorithm is optimized for high sensitivity while maintaining a low false
discovery rate by filtering out false positives from repetitive sequences and
mapping errors.
It employs BWA-MEM\supercite{li_fast_2009} for initial alignment, with
\gls{ciri2} distinguishing itself by its efficient use of computational
resources and ability to handle mixed read lengths, making it faster and more
memory-efficient compared to other methods\supercite{gao_circular_2018}.
