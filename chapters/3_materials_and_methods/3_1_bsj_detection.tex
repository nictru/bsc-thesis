\subsection{circRNA detection}
\label{subsec:circrna_detection}
What is the main task here?

\subsubsection{find\_circ (2013)\supercite{memczak_circular_2013}}
Find\_circ is one of the pioneering tools for circRNA detection, specifically
designed for identifying circular RNAs by leveraging RNA-seq data.
It employs a novel alignment strategy, splitting reads that do not map linearly
to the genome into smaller fragments, which are then re-aligned to detect
back-splice junctions (BSJs)\supercite{memczak_circular_2013}.
This tool does not rely on any known annotations and processes RNA-seq reads
using Bowtie, an efficient aligner for identifying BSJ reads.
One of its key features is that it filters for BSJ reads, removing those that
align entirely to the genome, which makes it stand out from methods relying
heavily on gene annotations\supercite{memczak_circular_2013}.

While it has been effective in many studies, it may not capture circRNAs with
non-canonical splice signals, which can limit its detection
capabilities\supercite{sekar_circular_2018,liu_prkra_2022}.
Nonetheless, find\_circ remains a popular choice for researchers due to its
ease of use and integration with RNA-seq workflows.

\subsubsection{Segemehl (2014)\supercite{hoffmann_multi-split_2014}}
Segemehl integrates a sensitive and flexible read-matching algorithm based on
enhanced suffix arrays (ESAs).
It identifies BSJ reads through split-read mapping, where the RNA-seq data is
aligned using a dynamic programming strategy that identifies splicing,
trans-splicing, and fusion transcripts.
Its major advantage over other tools is its ability to map reads containing
multiple split sites, which boosts sensitivity for detecting circRNAs even in
complex cases like long reads or sequences with
errors\supercite{hoffmann_multi-split_2014}.

While Segemehl has shown promise in identifying circRNAs, its performance can
vary depending on the specific dataset and experimental
conditions\supercite{gao_ciri_2015,zeng_comprehensive_2017}.

\subsubsection{DCC (2016)\supercite{cheng_specific_2016}}
DCC relies on STAR as its underlying aligner, which maps RNA-seq reads to the
genome.
It uses a combination of filters to detect BSJs by distinguishing back-splice
reads from linear splicing.
DCC integrates across replicates to minimize false positives and improve the
precision of circRNA detection.
What sets DCC apart is its post-mapping step, where it not only detects
circRNAs but also estimates their expression relative to the host gene using
read counts from junction and non-junction reads.
This makes it highly useful for comparing circRNA expression across
conditions\supercite{cheng_specific_2016}.
DCC has been validated in numerous studies, demonstrating high reliability and
accuracy in circRNA detection\supercite{paraboschi_interpreting_2018}.

\subsubsection{CIRCexplorer2 (2016)\supercite{zhang_diverse_2016}}
CIRCexplorer2 is another widely used tool that focuses on the detection of
circRNAs by analyzing RNA-seq data.
It employs a unique strategy that combines both back-splice junction reads and
linear RNA reads to improve circRNA identification.
CIRCexplorer2 has demonstrated robust performance in various studies, often
ranking high in comparative evaluations against other circRNA detection
tools\supercite{zeng_comprehensive_2017,nicolet_circular_2018}.
Its ability to provide detailed annotations and quantifications of circRNAs
makes it a valuable resource for researchers\supercite{hansen_comparison_2016}.

\subsubsection{CIRI2 (2018)\supercite{gao_circular_2018}}
CIRI2 improves upon the original CIRI\supercite{gao_ciri_2015} by using a
maximum likelihood estimation (MLE) based on multiple seed matching to detect
BSJ reads.
This algorithm is optimized for high sensitivity while maintaining a low false
discovery rate by filtering out false positives from repetitive sequences and
mapping errors.
It employs BWA-MEM for initial alignment, with CIRI2 distinguishing itself by
its efficient use of computational resources and ability to handle mixed read
lengths, making it faster and more memory-efficient compared to other
methods\supercite{gao_circular_2018}.
