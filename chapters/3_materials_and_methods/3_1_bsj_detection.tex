\subsection{circRNA detection}
\label{subsec:circrna_detection}
What is the main task here?

\subsubsection{CIRI2}
CIRI2 (CircRNA Identifier version 2) is an efficient algorithm designed for de
novo identification of circRNAs.
It utilizes a two-step approach: first, it identifies back-splice junction
reads, and then it filters these reads based on specific criteria to enhance
detection accuracy.
CIRI2 has been shown to outperform other tools in terms of sensitivity and
specificity, particularly in datasets with varying sequencing
depths\supercite{gao_ciri_2015,zheng_reconstruction_2019}.
This tool is particularly advantageous for large-scale circRNA studies due to
its unbiased nature and ability to handle complex
transcriptomes\supercite{chuang_assessing_2023}.

\subsubsection{CIRCexplorer2}
CIRCexplorer2 is another widely used tool that focuses on the detection of
circRNAs by analyzing RNA-seq data.
It employs a unique strategy that combines both back-splice junction reads and
linear RNA reads to improve circRNA identification.
CIRCexplorer2 has demonstrated robust performance in various studies, often
ranking high in comparative evaluations against other circRNA detection
tools\supercite{zeng_comprehensive_2017,nicolet_circular_2018}.
Its ability to provide detailed annotations and quantifications of circRNAs
makes it a valuable resource for researchers\supercite{hansen_comparison_2016}.

\subsubsection{circRNA finder}
The circRNA finder is a tool specifically designed for the identification of
circRNAs from RNA-seq data.
It employs a read mapping strategy that focuses on detecting chimeric reads,
which are indicative of circRNA formation.
This tool has been effectively utilized in various studies to identify circRNAs
across different biological contexts, although it may have limitations in
detecting circRNAs with non-canonical splice
signals\supercite{sekar_circular_2018,liu_prkra_2022}.
Its straightforward approach makes it accessible for researchers new to circRNA
analysis.

\subsubsection{DCC}
DCC (DCC: Detecting Circular RNAs) is a versatile software that allows for the
detection and quantification of circRNAs from RNA-seq data.
It computes expression levels of circRNAs independently of their linear
counterparts, which is crucial for understanding the functional roles of
circRNAs in various biological
processes\supercite{jakobi_profiling_2016,man_profiling_2020}.
DCC has been validated in numerous studies, demonstrating high reliability and
accuracy in circRNA detection\supercite{paraboschi_interpreting_2018}.
Its ability to integrate with other bioinformatics tools further enhances its
utility in comprehensive circRNA analyses.

\subsubsection{find\_circ}
Find\_circ is a widely recognized tool that identifies circRNAs by focusing on
back-splice junctions and employing a filtering strategy based on splice
signals.
While it has been effective in many studies, it may not capture circRNAs with
non-canonical splice signals, which can limit its detection
capabilities\supercite{sekar_circular_2018,liu_prkra_2022}.
Nonetheless, find\_circ remains a popular choice for researchers due to its
ease of use and integration with RNA-seq workflows.

\subsubsection{MapSplice}
MapSplice is primarily known for its role in splicing analysis but has also
been adapted for circRNA detection.
It utilizes a splice-aware alignment strategy to identify back-splice
junctions, making it suitable for circRNA studies.
However, its performance in circRNA detection may not be as robust as dedicated
circRNA tools like CIRI2 or
DCC\supercite{zeng_comprehensive_2017,chuang_nclscan_2016}.
MapSplice's strength lies in its ability to handle complex splicing events,
which can be beneficial in certain research contexts.

\subsubsection{Segemehl}
Segemehl is another tool that has been employed for circRNA detection,
particularly in studies involving complex transcriptomes.
It utilizes a unique alignment strategy that allows for the detection of both
linear and circular transcripts.
While Segemehl has shown promise in identifying circRNAs, its performance can
vary depending on the specific dataset and experimental
conditions\supercite{gao_ciri_2015,zeng_comprehensive_2017}.
Its flexibility in handling different types of RNA-seq data makes it a valuable
option for researchers exploring circRNA biology.