\subsection{\gls{mirna} interaction analysis}
The analysis of \glspl{crna} interactions with \glspl{mirna} is crucial for
understanding the regulatory roles of \glspl{crna} in various biological
processes.
Two prominent tools used for predicting \gls{crna}-\gls{mirna} interactions are
MiRanda and TargetScan.
These tools leverage sequence complementarity and binding energy calculations
to identify potential \gls{mirna} binding sites within \gls{crna} sequences,
thereby elucidating their roles as \glspl{cerna}.

\subsubsection{MiRanda}
MiRanda is a widely utilized algorithm that predicts \gls{mirna} targets based
on sequence complementarity and the stability of the \gls{rna} duplex formed
between the \gls{mirna} and its target.
It has been effectively employed in various studies to analyze
\gls{crna}-\gls{mirna} interactions.
For instance, Vromman et al.
noted that
MiRanda is often used alongside TargetScan to predict \gls{mirna} binding sites
in
\gls{crna} sequences, contributing to a better understanding of \gls{crna}
functions
in gene regulation\supercite{vromman_closing_2021}.
Similarly, Zhang et al.
utilized
MiRanda in conjunction with TargetScan to predict \glspl{mre} in differentially
expressed \glspl{crna}, demonstrating its utility in
identifying significant interactions\supercite{zhang_microarray_2017}.

\subsubsection{TargetScan}
TargetScan, on the other hand, focuses on the identification of conserved
\gls{mirna} binding sites across species, which enhances the reliability of the
predicted interactions.
This tool has been integrated into various studies to explore the regulatory
networks involving \glspl{crna} and \glspl{mirna}.
For example, Jin et al.
employed MiRanda and TargetScan to predict interactions between \glspl{crna}
and
\glspl{mirna}, reinforcing the hypothesis that \glspl{crna} can act as
\gls{mirna}
sponges\supercite{jin_changes_2018}.
Furthermore, the combination of these tools allows researchers to construct
comprehensive \gls{crna}-\gls{mirna}-\gls{mrna} networks, elucidating the
complex regulatory mechanisms at play in various
diseases\supercite{he_construction_2021,zhang_construction_2021}.
