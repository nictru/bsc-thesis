\subsection{miRNA interaction analysis}
The analysis of circular RNA (\gls{crna}) interactions with microRNAs (miRNAs)
is crucial for understanding the regulatory roles of \gls{crna}s in various
biological processes.
Two prominent tools used for predicting \gls{crna}-miRNA interactions are
MiRanda and TargetScan.
These tools leverage sequence complementarity and binding energy calculations
to identify potential miRNA binding sites within \gls{crna} sequences, thereby
elucidating their roles as competitive endogenous RNAs (ceRNAs).

\subsubsection{MiRanda}
MiRanda is a widely utilized algorithm that predicts miRNA targets based on
sequence complementarity and the stability of the RNA duplex formed between the
miRNA and its target.
It has been effectively employed in various studies to analyze \gls{crna}-miRNA
interactions.
For instance, Vromman et al.
noted that
MiRanda is often used alongside TargetScan to predict miRNA binding sites in
\gls{crna} sequences, contributing to a better understanding of \gls{crna}
functions
in gene regulation\supercite{vromman_closing_2021}.
Similarly, Zhang et al.
utilized
MiRanda in conjunction with TargetScan to predict microRNA response elements
(MREs) in differentially expressed \gls{crna}s, demonstrating its utility in
identifying significant interactions\supercite{zhang_microarray_2017}.

\subsubsection{TargetScan}
TargetScan, on the other hand, focuses on the identification of conserved miRNA
binding sites across species, which enhances the reliability of the predicted
interactions.
This tool has been integrated into various studies to explore the regulatory
networks involving \gls{crna}s and miRNAs.
For example, Jin et al.
employed MiRanda and TargetScan to predict interactions between \gls{crna}s and
miRNAs, reinforcing the hypothesis that \gls{crna}s can act as miRNA
sponges\supercite{jin_changes_2018}.
Furthermore, the combination of these tools allows researchers to construct
comprehensive \gls{crna}-miRNA-mRNA networks, elucidating the complex
regulatory mechanisms at play in various
diseases\supercite{he_construction_2021,zhang_construction_2021}.
