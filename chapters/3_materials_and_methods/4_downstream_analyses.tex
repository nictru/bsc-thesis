\section{Downstream analyses}

\subsection{Differential expression analysis}

\subsubsection{DESeq2}

DESeq2 is a widely utilized R package designed for analyzing differential gene
expression from RNA-seq data, particularly through the application of the
negative binomial distribution to model count data.
It provides robust normalization methods and statistical tests to identify
differentially expressed genes and \glspl{crna} across various biological
conditions\supercite{love_moderated_2014,hu_integrative_2019,iparraguirre_blood_2023}.
Recent studies have effectively employed DESeq2 for differential circRNA
expression analysis, revealing significant changes in circRNA profiles under
different experimental conditions.
For instance, in the context of Varicella Zoster Virus infection, DESeq2
identified 532 upregulated and 253 downregulated circRNAs in infected
cells\supercite{yang_identification_2024}.
Similarly, other research has confirmed its effectiveness in analyzing circRNA
expression, highlighting its role in identifying potential biomarkers and
understanding regulatory mechanisms in various
diseases\supercite{wang_characterization_2022,feng_circrna_2022,feng_comprehensive_2022}.

\subsubsection{\Glsfmtlong{ciriquant}}

While \gls{ciriquant} has already been introduced in \cref{sec:ciriquant} as a
quantification tool, it also provides a differential expression analysis
feature.
It employs a generalized fold-change method, which accounts for both the
magnitude of expression change and the variance in the posterior distribution
of log2 fold changes, thus providing a robust framework for identifying
differentially expressed \glspl{crna}\supercite{zhang_accurate_2020}.
While it lacks the flexibility of DESeq2 (e.g., in terms of design matrix
specification and tests for association with numeric covariates), it was
specifically designed for \gls{crna} data.

\subsection{\Glsfmtfull{fea}}
\glsunset{fea}

\Gls{fea} is a crucial bioinformatics approach used to
interpret large-scale genomic data by identifying biological functions and
pathways associated with a set of genes.
This method allows researchers to discern the biological significance of gene
lists derived from high-throughput experiments, such as \gls{rna-seq} or
microarray studies, by comparing them against predefined gene sets from
databases like \gls{go} and
\gls{kegg}\supercite{du_kegg-path_2014,wu_clusterprofiler_2021}.
Enrichment analysis helps to uncover underlying biological processes, molecular
functions, and cellular components that are overrepresented in the gene sets,
thereby facilitating a deeper understanding of the molecular mechanisms at play
in various biological contexts\supercite{wu_clusterprofiler_2021}.

\subsubsection{\Glsfmtfull{go}}
\glsunset{go}

\Gls{go} provides a structured vocabulary that describes gene
functions across different organisms, encompassing three main domains:
biological processes, molecular functions, and cellular
components\supercite{ashburner_gene_2000}.
The \gls{go} framework enables researchers to annotate genes consistently and
to compare gene functions across species, which is particularly valuable in
evolutionary biology and comparative genomics\supercite{ashburner_gene_2000}.
The dynamic nature of \gls{go} allows for the continuous updating of
annotations as new biological knowledge emerges, thus supporting ongoing
research efforts in functional genomics\supercite{ashburner_gene_2000}.
This comprehensive approach to gene annotation is essential for elucidating the
roles of genes in complex biological systems and diseases.

\subsubsection{ClusterProfiler}

The ClusterProfiler R package is a widely used tool for conducting functional
enrichment analysis, particularly for \gls{go} and \gls{kegg}
pathways\supercite{wu_clusterprofiler_2021,yu_clusterprofiler_2012}.
It offers a user-friendly interface for analyzing and visualizing the results
of enrichment analyses, making it accessible for researchers with varying
levels of bioinformatics expertise\supercite{dai_identification_2020}.
ClusterProfiler not only supports enrichment analysis but also facilitates the
comparison of biological themes among gene clusters, enhancing the
interpretability of high-throughput
data\supercite{yu_clusterprofiler_2012,chen_identification_2021}.
Its integration with other R packages allows for comprehensive data analysis,
including visualization of results through various graphical representations
\supercite{wu_clusterprofiler_2021,xie_identification_2022}.
This versatility makes clusterProfiler an invaluable resource in the field of
genomics and systems biology.
