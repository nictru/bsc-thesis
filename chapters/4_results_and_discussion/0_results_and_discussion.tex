\chapter{Results and Discussion}

From a molecular perspective, circRNAs are similar to their linear
counterparts, with the primary distinction being their circular structure.
As explained in \cref{sec:circrna_biogenesis}, circRNAs are derived from a
linear precursor (identical to linear RNA) that undergoes back-splicing to form
a loop.
Thus, the only way to computationally identify circRNAs is by detecting reads
spanning the back-splice junction (BSJ), which cannot be accounted for by
canonical forward splicing.
Several tools for BSJ detection are discussed in
\cref{subsec:circrna_detection}.

However, detecting BSJs only pinpoints the start and end of a circRNA within
the genome, without revealing its internal structure.
To reconstruct the full circRNA sequence, a de novo assembly of the circular
transcriptome that accounts for back-splice junctions is required.
Tools like CIRI-full, circRNAfull, and JCcirc are capable of this task, but
they rely on paired-end or long-read sequencing data.
To date, no tools are available for de novo assembly of circRNAs using
single-end short-read data.
Since the data used in this thesis is single-end, the focus is limited to the
detection, quantification, and analysis of BSJs.

\section{\Acrfull{bsj} detection}

For detection of \gls{bsj}s, I used the five tools already introduced in
\cref{subsec:circrna_detection}.
As shown in \cref{fig:detection_bars}, find\_circ, CIRI2, DCC, and
circexplorer2 detect a similar number of \gls{bsj}s, while segemehl detects
almost ten times as many \gls{bsj}s as its closest competitor, DCC.

\begin{figure}[ht] \centering

    \includegraphics[width=0.6\textwidth]{chapters/4_results_and_discussion/figures/detection/n_bsjs_detected.png}
    \caption{Number of \gls{bsj}s detected by each tool.
        While find\_circ, CIRI2, DCC and circexplorer2 detect a similar number of
        \gls{bsj}s, segemehl detects a much larger number of \gls{bsj}s.
    }
    \label{fig:detection_bars}
\end{figure}
Similar behavior was previously observed by \textcite{zeng_comprehensive_2017},
where segemehl was among the top performers in terms of sensitivity, but also
had a high false positive rate.
The lowest numbers of \gls{bsj}s were detected by circexplorer2 and CIRI2,
which both have built-in filters to reduce false
positives\supercite{zhang_diverse_2016,gao_circular_2018}.

\subsection{Agreement between tools and the role of \textit{max shift}}

Although tools like CIRI2 and CircExplorer2 have shown good performance in
benchmarks, having multiple tools agree on the same \gls{bsj} can be a good
indicator of the reliability of the detection.

To assess the agreement between the tools, I used UpSet plots, which show the
overlap of \gls{bsj}s detected by different tools.
When identifying the overlap between tools, the most strict approach is to
consider only \gls{bsj}s with identical start and end positions and on the same
strand as the same \gls{bsj}.
The according plot is shown in \cref{fig:detection_upset_0}.
While there are a total of x \gls{bsj}s detected by at least two tools, only y
\gls{bsj}s are detected by three tools, and none are detected by four or five
tools.

\begin{figure}[ht]
    \centering

    \includegraphics[width=\textwidth]{chapters/4_results_and_discussion/figures/detection/upset/shift_0.png}
    \caption{Upset plot illustrating the overlap between \gls{bsj}s detected by
        different tools.
        Only \gls{bsj}s with identical start and end positions are considered the same.
        Only combinations with at least 10 common \gls{bsj}s are shown.
    }
    \label{fig:detection_upset_0}
\end{figure}

As this agreement is relatively low, I investigated the detected \gls{bsj}s
more closely and found that tools often have very similar \gls{bsj}s, but with
slight differences in the start and end positions.

In order to quantify how frequently the slight mismatches occur, I introduced a
\textit{max shift} parameter.
With this parameter, if tool A detects a \gls{bsj} and tool B detects another
\gls{bsj}, where the start and end positions differ by at most \textit{max
    shift} nucleotides, both \gls{bsj}s are considered to be supported by both
tools.
A more detailed explanation is given in \cref{fig:detection_shift_schematic}.

\begin{figure}[ht] \centering

    \includegraphics[width=0.7\textwidth]{chapters/4_results_and_discussion/figures/grouping.png}
    \caption{Schematic illustrating two different scenarios of \gls{bsj}
        matches
        across tools.
        In the green scenario, the \gls{bsj}s are exactly the same.
        This is what is required in order to be counted as a match in
        \cref{fig:detection_upset_0}.
        However, this rarely occurs in practice.
        More frequently, the scenario illustrated in blue occurs.
        Here, tools detect very similar \gls{bsj}s, with only a few nucleotides
        difference.
        In order to account for this, a \textit{max shift} parameter is introduced.
        While in the blue scenario, all \gls{bsj}s would only be supported by one tool,
        with a \textit{max shift} of 1, the result would change drastically.
        The \gls{bsj} marked as A would be supported by two others (B and D), similarly
        B would be supported by two others (A and D).
        C would only be supported by D, as both the ends of both A and B differ by more
        than 1.
        D would be supported by all others.
    }
    \label{fig:detection_shift_schematic}
\end{figure}

In order to quantify the magnitude of the effect of the \textit{max shift}
parameter, I calculated the level of agreement between tools for different
values of \textit{max shift}.
The results are shown in \cref{fig:shift_agreement}.

\begin{figure}[ht]
    \centering

    \includegraphics[width=0.7\textwidth]{chapters/4_results_and_discussion/figures/detection/shift_agreement.png}
    \caption{Stacked bar plot showing the level of agreement between tools for
        different values of the \textit{max shift} parameter.
        While the distribution changes drastically when increasing the \textit{max
            shift} parameter from 0 to 1, the distribution remains relatively stable for
        higher values.
        While the number of \gls{bsj}s detected by 2-3 tools continues to increase when
        raising the \textit{max shift} parameter to 20 or 50, the number of \gls{bsj}s
        detected by 4-5 tools remains nearly constant.
    }
    \label{fig:shift_agreement}
\end{figure}

As shown in \cref{fig:shift_agreement}, the distribution of the number of tools
supporting a \gls{bsj} changes drastically when increasing the \textit{max
    shift} parameter from 0 to 1.
However, the distribution remains relatively stable, especially for agreement
between four or five tools.
Increasing the \textit{max shift} parameter to large values does have an effect
on the number of \gls{bsj}s detected by two or three tools, but this comes with
the risk of increasing the number of false positives, since this way \gls{bsj}s
with a relatively large difference in start and end positions are still
considered as evidence for each other.

For the following analyses, I chose a \textit{max shift} parameter of 3 and a
minimum agreement of 4 tools, as this appeared to be a good compromise between
sensitivity and specificity.
The resulting UpSet plot is shown in \cref{fig:detection_upset_3}.

\begin{figure}[ht]
    \centering

    \includegraphics[width=\textwidth]{chapters/4_results_and_discussion/figures/detection/upset/shift_3.png}
    \caption{Upset plot illustrating the overlap between \gls{bsj}s detected by
        different tools.
        Agreement was calculated using a \textit{max shift} of 3.
        Only combinations including at least 2 tools and 10 common \gls{bsj}s are
        shown.
    }
    \label{fig:detection_upset_3}
\end{figure}

\section{Quantification}

As mentioned in \cref{sec:crna_quantification}, all the \gls{bsj} detection
tools also quantify the number of reads supporting each \gls{bsj}.
However, there are several tools that focus on the quantification of
\gls{crna}s based on previously detected \gls{bsj}s.
nf-core/circrna offers two such tools:
CIRIquant
and psirc-quant.

In the following, I will compare the quantification results obtained by
the detection tools to those obtained by CIRIquant and psirc-quant.

\begin{figure}[ht]
    \centering

    \includegraphics[width=0.7\textwidth]{chapters/4_results_and_discussion/figures/quantification/correlation_heatmap.png}
    \caption{Correlation heatmap
    }
    \label{fig:quantification_correlation_heatmap}
\end{figure}
\section{Differential expression analysis}
Differential expression analysis was performed using the methods described in
\cref{sec:material_dea}: DESeq2 and \gls{ciriquant}.
In the following sections of this chapter, the results of the differential
expression analysis are presented.

\subsection{\Glspl{crna} associated with estrogen receptor expression}

In order to identify potential markers of estrogen levels, the expression of the


\begin{figure}[H] \begin{tabular}{cc} \begin{subfigure}{0.5\textwidth}
                 \centering

                 \includegraphics[width=\linewidth]{chapters/4_results_and_discussion/figures/dea/deseq2/esr1/volcano.png}
                 \caption{Volcano plot based on differential expression
                     analysis using DESeq2.
                     Coloring is based on the presence of a supporting database entry in any of the
                     used databases.
                 } \label{fig:esr1_volcano} \end{subfigure}
        \begin{subfigure}{0.5\textwidth} \centering

            \includegraphics[width=\linewidth]{chapters/4_results_and_discussion/figures/dea/deseq2/esr1/dot.png}
            \caption{Functional enrichment of the host genes of the
                \glspl{crna} marked as significantly associated with \gls{esr1}
                expression.
            }

            \label{fig:esr1_go_terms}
        \end{subfigure}
    \end{tabular}
    \caption{Results of differential expression analysis based on the
        association with \gls{esr1} expression (numeric contrast).
        Expression of the \gls{esr1} gene was determined using the nf-core/rnaseq
        pipeline\supercite{patel_nf-corernaseq_2024}.
        The following design formula was used: $\sim age + transgene + induction + drug
            + esr1$.
        The expression matrix containing all samples was tested for log2 fold changes
        with greater absolute values than 2.
        Benjamini-Hochberg correction\supercite{benjamini_controlling_1995} was used to
        adjust the $p$-values for multiple testing.
        \Glspl{crna} were considered significantly associated with \gls{esr1}
        expression if they had an adjusted $p$-value of less than 0.05.
    }
    \label{fig:dea_esr1}
\end{figure}

\subsection{\Glspl{crna} associated with tamoxifen treatment}

\begin{figure}[H] \begin{tabular}{cc}
        \begin{subfigure}{0.5\textwidth} \centering

            \includegraphics[width=\linewidth]{chapters/4_results_and_discussion/figures/dea/deseq2/tamoxifen/volcano.png}
            \caption{Volcano plot illustrating the differential expression
                results obtained using DESeq2.
                The following design formula was used: $\sim age + transgene + induction +
                    drug$.
                The expression matrix was tested for log2 fold changes with greater absolute
                values than 2.
            }
            \label{fig:tamoxifen_volcano_deseq2}
        \end{subfigure}
        \begin{subfigure}{0.5\textwidth}
            \centering

            \includegraphics[width=\linewidth]{chapters/4_results_and_discussion/figures/dea/ciriquant/tamoxifen/volcano.png}
            \caption{Volcano plot illustrating the differential expression
                results obtained using \gls{ciriquant}.
                Samples without treatment were used as the control group, and samples treated
                with tamoxifen were used as the treatment group.
            }
            \label{fig:tamoxifen_volcano_ciriquant}
        \end{subfigure}

    \end{tabular}
    \caption{Results of differential expression analysis comparing untreated
        samples with samples treated using tamoxifen (binary contrast).
        Benjamini-Hochberg correction\supercite{benjamini_controlling_1995} was used to
        adjust the $p$-values for multiple testing.
        \Glspl{crna} were considered significantly associated with tamoxifen
        treatment if they had an adjusted $p$-value of less than 0.05.
    } \label{fig:tamoxifen_volcano} \end{figure}

\subsubsection{Potential target genes}

\subsection{\Glspl{crna} associated with letrozole treatment}

\begin{figure}[H] \begin{tabular}{cc} \begin{subfigure}{0.5\textwidth}
                 \centering

                 \includegraphics[width=\linewidth]{chapters/4_results_and_discussion/figures/dea/deseq2/letrozole/volcano.png}
                 \caption{Volcano plot illustrating the differential expression
                     results obtained using DESeq2.
                     The same setup as in \cref{fig:tamoxifen_volcano_deseq2} was used.
                 }
                 \label{fig:letrozole_volcano_deseq2}
             \end{subfigure}
        \begin{subfigure}{0.5\textwidth}
            \centering

            \includegraphics[width=\linewidth]{chapters/4_results_and_discussion/figures/dea/ciriquant/letrozole/volcano.png}
            \caption{Volcano plot illustrating the differential expression
                results obtained using \gls{ciriquant}.
                Samples without treatment were used as the control group, and samples treated
                with letrozole were used as the treatment group.
            }
            \label{fig:letrozole_volcano_ciriquant}
        \end{subfigure} &

    \end{tabular}
    \caption{Results of differential expression analysis using the same
        approach as in
        \cref{fig:tamoxifen_volcano}, but comparing samples treated with
        letrozole
        to untreated samples.
    }
    \label{fig:letrozole_volcano}
\end{figure}

\subsubsection{Potential target genes}

\include{chapters/4_results_and_discussion/4_interpretation.tex}
