\section{\Acrfull{bsj} detection}

For detection of \gls{bsj}s, five different detection tools were used:
circexplorer2, CIRI2, DCC, find\_circ and segemehl.
While the nf-core/circrna pipeline provides a wider selection of tools, these
are the only ones which work with single-end sequencing data.

\begin{figure}[ht]
    \centering

    \includegraphics[width=0.7\textwidth]{chapters/4_results_and_discussion/figures/detection/n_bsjs_detected.png}
    \caption{Number of \gls{bsj}s detected by each tool.
        While find\_circ, CIRI2, DCC and circexplorer2 detect a similar number of
        \gls{bsj}s, segemehl detects a much larger number of \gls{bsj}s.
    }
    \label{fig:detection_bars}
\end{figure}

\begin{figure}[ht]
    \begin{tabular}{cc}
        \begin{subfigure}{.5\textwidth}
            \centering

            \includegraphics[width=\linewidth]{chapters/4_results_and_discussion/figures/detection/upset/diff_0_strand.png}
            \caption{No shift allowed, strand considered}
            \label{fig:detection_upset_full}
        \end{subfigure}
         &
        \begin{subfigure}{.5\textwidth}
            \centering

            \includegraphics[width=\linewidth]{chapters/4_results_and_discussion/figures/detection/upset/diff_0_nostrand.png}
            \caption{No shift allowed, strand ignored}
            \label{fig:detection_upset_nostrand}
        \end{subfigure} \\
        \multicolumn{2}{c}{
            \begin{subfigure}{\textwidth}
                \centering

                \includegraphics[width=\linewidth]{chapters/4_results_and_discussion/figures/detection/upset/diff_1_nostrand.png}
                \caption{Shift of 1 allowed, strand ignored}
                \label{fig:detection_upset_microshift}

            \end{subfigure}}
    \end{tabular}
    \caption{Upset plots showing the overlap of \gls{bsj}s detected by
        different
        tools using various grouping criteria.
        Only groups with at least two agreeing tools are displayed.
        When considering strand information and only counting \gls{bsj}s with identical
        start and end positions, many \gls{bsj}s are detected by just two tools, with
        very few detected by three tools (\cref{fig:detection_upset_full}).
        Ignoring the strand information substantially increases the number of
        \gls{bsj}s detected by 3 tools (\cref{fig:detection_upset_nostrand}).
        Allowing a shift of 1 bp in the start and end positions while ignoring the
        strand information changes the results drastically, leading to a large number
        of circRNAs with agreement between 4 and even 5 tools
        (\cref{fig:detection_upset_microshift}).
    }
    \label{fig:detection_upset}
\end{figure}

\begin{figure}[ht]
    \centering

    \includegraphics[width=0.7\textwidth]{chapters/4_results_and_discussion/figures/detection/pies.png}
    \caption{Pie charts showing the agreement of each tool with the others.
        The inner circle shows the agreement without allowing a shift in the start and
        end positions, while the outer circle shows the agreement when allowing a shift
        of 1 bp.
        While find\_circ, dcc, circexplorer2, and ciriquant behave similarly, segemehl
        detects a much larger number of \gls{bsj}s, with a large portion of them not
        being detected by any other tool.
    }
    \label{fig:detection_pies}
\end{figure}

\begin{figure}[ht]
    \begin{tabular}{cc}
        \begin{subfigure}{.4\textwidth}
            \centering

            \includegraphics[width=\linewidth]{chapters/4_results_and_discussion/figures/detection/density/circexplorer2.png}
            \caption{CircExplorer2}
            \label{fig:detection_density_circexplorer2}
        \end{subfigure}
         &
        \begin{subfigure}{.4\textwidth}
            \centering

            \includegraphics[width=\linewidth]{chapters/4_results_and_discussion/figures/detection/density/ciri2.png}
            \caption{CIRI2}
            \label{fig:detection_density_ciri2}
        \end{subfigure}     \\
        \begin{subfigure}{.4\textwidth}
            \centering

            \includegraphics[width=\linewidth]{chapters/4_results_and_discussion/figures/detection/density/dcc.png}
            \caption{DCC}
            \label{fig:detection_density_dcc}
        \end{subfigure}
         &
        \begin{subfigure}{.4\textwidth}
            \centering

            \includegraphics[width=\linewidth]{chapters/4_results_and_discussion/figures/detection/density/find_circ.png}
            \caption{find\_circ}
            \label{fig:detection_density_find-circ}
        \end{subfigure} \\
        \begin{subfigure}{.4\textwidth}
            \centering

            \includegraphics[width=\linewidth]{chapters/4_results_and_discussion/figures/detection/density/segemehl.png}
            \caption{Segemehl}
            \label{fig:detection_density_segemehl}
        \end{subfigure}
         &
        \begin{subfigure}{.4\textwidth}
            \centering
            % TODO: Add legend
            %\includegraphics[width=\linewidth]{chapters/4_results_and_discussion/figures/detection/density/find_circ.png}
            \caption{Legend}
        \end{subfigure}
    \end{tabular}
    \caption{Distributions of \gls{bsj}s detected by each tool.
        The x-axis shows the number of tools that detected a \gls{bsj}, while the
        y-axis shows the logarithm of the number of reads supporting the \gls{bsj}
        across all samples.
        The color indicates the density of \gls{bsj}s at a given point.
        The stacked bar plots above and left of the scatter plots show the number of
        agreeing tools in the respective slice.
    }
    \label{fig:detection_density}
\end{figure}
