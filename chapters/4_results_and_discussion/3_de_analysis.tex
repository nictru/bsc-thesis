\section{Differential expression analysis}

\subsection{\Glspl{crna} associated with estrogen receptor expression}

\begin{figure}[H] \begin{tabular}{cc}
        \begin{subfigure}{0.5\textwidth} \centering

            \includegraphics[width=\linewidth]{chapters/4_results_and_discussion/figures/dea/deseq2/esr1/volcano.png}
            \caption{Volcano plot based on differential expression analysis
                using DESeq2.
                Coloring is based on the presence of a supporting database entry in any of the
                used databases.
            } \label{fig:esr1_volcano} \end{subfigure}
        \begin{subfigure}{0.5\textwidth} \centering

            \includegraphics[width=\linewidth]{chapters/4_results_and_discussion/figures/dea/deseq2/esr1/dot.png}
            \caption{Functional enrichment of the host genes of the
                \glspl{crna} marked as significantly associated with Esr1
                expression.
            }

            \label{fig:esr1_go_terms}
        \end{subfigure}
    \end{tabular}
    \caption{Results of differential expression analysis based on the
        association with Esr1 expression.
        Expression of the Esr1 gene was determined using the nf-core/rnaseq
        pipeline\supercite{patel_nf-corernaseq_2024}.
        The following design formula was used: $\sim age + transgene + induction + drug
            + esr1$.
        The expression matrix was tested for log2 fold changes with greater absolute
        values than 2.
        Benjamini-Hochberg correction was used to adjust the $p$-values for multiple
        testing.
        \Glspl{crna} were considered significantly associated with Esr1
        expression if they had an adjusted $p$-value of less than 0.05.
    }
    \label{fig:dea_esr1}
\end{figure}

\subsection{\Glspl{crna} associated with tamoxifen induction}

\begin{figure}[H] \begin{tabular}{cc}
        \begin{subfigure}{0.5\textwidth} \centering

            \includegraphics[width=\linewidth]{chapters/4_results_and_discussion/figures/dea/deseq2/tamoxifen/volcano.png}
            \caption{Hello world!
            }
            \label{fig:tamoxifen_volcano_deseq2}
        \end{subfigure}
        \begin{subfigure}{0.5\textwidth}
            \centering

            \includegraphics[width=\linewidth]{chapters/4_results_and_discussion/figures/dea/ciriquant/tamoxifen/volcano.png}
            \caption{Hello world!
            }
            \label{fig:tamoxifen_volcano_ciriquant}
        \end{subfigure} &

    \end{tabular}
    \caption{Hello world!
    }
    \label{fig:tamoxifen_volcano}
\end{figure}

\subsection{\Glspl{crna} associated with letrozole induction}

\begin{figure}[H] \begin{tabular}{cc}
        \begin{subfigure}{0.5\textwidth} \centering

            \includegraphics[width=\linewidth]{chapters/4_results_and_discussion/figures/dea/deseq2/letrozole/volcano.png}
            \caption{Hello world!
            }
            \label{fig:letrozole_volcano_deseq2}
        \end{subfigure}
        \begin{subfigure}{0.5\textwidth}
            \centering

            \includegraphics[width=\linewidth]{chapters/4_results_and_discussion/figures/dea/ciriquant/letrozole/volcano.png}
            \caption{Hello world!
            }
            \label{fig:letrozole_volcano_ciriquant}
        \end{subfigure} &

    \end{tabular}
    \caption{Hello world!
    }
    \label{fig:letrozole_volcano}
\end{figure}

\subsection{Known \glsfmtshortpl{snp} located in differentially expressed
    \glsfmtshortpl{crna}}
