\chapter{Conclusion and Outlook}
\label{sec:conclusion_and_outlook}

In this thesis, I utilized the \gls{nf-circrna} pipeline to analyze the
regulatory roles of \glspl{crna} in estrogen signaling and breast cancer.
Despite the limitations of the datasets for comprehensive \glspl{crna}
analysis, the findings provide valuable insights and highlight novel
methodological contributions that could advance future research.

One of the key technical challenges encountered was the slight positional
variation of detected \glspl{bsj} across different tools and samples.
To address this, I introduced the \textit{max shift} parameter, which
algorithmically mitigates these discrepancies, significantly improving the
consistency of \gls{crna} detection between tools.
While full validation of the \textit{max shift} parameter is yet to be
completed, this approach offers a promising solution that could be widely
adopted in future \gls{crna} studies to enhance detection robustness.

In terms of quantification, I found that summing the counts derived from
multiple detection tools reduced the variance between them, yielding smoother
results.
Dedicated quantification tools like \gls{ciriquant} and \gls{psirc-quant} were
tested, but did not deliver promising results.
This is most likely due to the fact that the data used in this study was not
optimal for the analysis of \glspl{crna}.
This emphasizes the importance of selecting appropriate data for such studies
and suggests that further optimization of the quantification tools may be
needed.

The differential expression analysis uncovered a substantial number of
\glspl{crna} significantly associated with \gls{esr1} expression and treatment
with \gls{let} and \gls{tam}.
Notably, three \glspl{crna} stood out due to their high significance, and two
of these were absent from \gls{circatlas}, suggesting they may be novel
\glspl{crna} related to estrogen signaling and breast cancer.
These \glspl{crna} could potentially serve as biomarkers for treatment efficacy
or provide insights into the molecular mechanisms underlying therapeutic
responses.

The exploration of \gls{mirna}-gene interactions further revealed that the
target genes of the identified candidate \glspl{crna} are involved in critical
biological processes and pathways associated with the effects of \gls{let} and
\gls{tam} treatment.
This demonstrates the regulatory potential of these \glspl{crna} in mediating
the response to breast cancer therapies.

\textbf{Outlook:}
Future research should prioritize the systematic evaluation of the \textit{max
    shift} parameter across different datasets and detection tools, with the goal
of determining an optimal value that maximizes \gls{crna} detection accuracy.
Additionally, further development of quantification methods tailored to
\gls{crna}-specific challenges is necessary.

Moreover, the identified candidate \glspl{crna} should be validated using more
comprehensive datasets and experimental approaches.
Functional assays could provide a deeper understanding of their roles in
estrogen signaling and their potential as biomarkers for therapeutic outcomes.
Ultimately, this line of research could contribute to more personalized breast
cancer treatment strategies by leveraging \glspl{crna} as diagnostic and
prognostic tools.
