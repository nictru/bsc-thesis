\chapter{Conclusion and Outlook}

In this thesis, I used the \gls{nf-circrna} pipeline to analyze the regulatory
roles of \glspl{crna} in estrogen signaling and breast cancer.
Although the used datasets were not optimal for the investigation of
\glspl{crna}, some interesting findings were obtained.

Looking at the exact positions of detected \glspl{bsj} in the genome revealed
that there are small shifts in the \gls{bsj} positions across different tools
and samples.
By introducing a \textit{max shift} parameter, I was able to algorithmically
overcome this issue and increase the overall agreement between the tools.
While proper validation of the \textit{max shift} parameter is still pending,
it provides a promising approach for future studies.

The analysis of different quantification methods showed that calculating the
sum of the counts derived from the detection tools smoothens out the
differences between the tools.
Dedicated quantification tools like \gls{ciriquant} and \gls{psirc-quant} were
tested, but did not deliver promising results.
This is most likely due to the fact that the data used in this study was not
optimal for the analysis of \glspl{crna}.

The differential expression analysis revealed large numbers of \glspl{crna}
significantly associated with \gls{esr1} expression and treatment with
\gls{let} and \gls{tam}.
While most of these \glspl{crna} formed clusters in the volcano plots, there
were three \glspl{crna} that stood out due to their high significance.
Two of them were not present in \gls{circatlas}, indicating that they might be
novel \glspl{crna} associated with estrogen signaling and breast cancer.

Target genes were obtained using a study that identified \gls{mirna}-gene
interactions in mouse mammary tissue.
The \gls{nf-circrna} pipeline was used to predict interactions between
differentially expressed \glspl{crna} and the \glspl{mirna} that are known to
be active in the tissue.
The results showed that the target genes of the differentially expressed
\glspl{crna} are involved in various biological processes and pathways related
to the effects of \gls{let} and \gls{tam} treatment.

Future work should focus on systematically investigating the effects of the
\textit{max shift} parameter on the results of the \gls{bsj} detection,
potentially leading to an optimal value for this parameter.

Furthermore, the discovered candidate \glspl{crna} should be validated using
more data and experimental methods.
