\section{Breast cancer}
Breast cancer is a heterogeneous group of malignancies that primarily originates
in the breast tissue, characterized by the uncontrolled growth of cells. It is
the most prevalent cancer among women globally, accounting for approximately
25.2\% of all cancer cases in females and being the second leading cause of
cancer-related deaths among women\supercite{pace_breast_2016}. The disease can
manifest in various forms, with distinct biological behaviors and responses to
treatment. Notably, breast cancer can be classified based on the presence of
hormone receptors, such as estrogen and progesterone receptors, and the human
epidermal growth factor receptor 2 (HER2)\supercite{eccles_critical_2013}. The
understanding of these subtypes is crucial, as they significantly influence
prognosis and treatment strategies.

\subsection{Causes and risk factors}

- Genetics
    - BRCA1/2 mutations
    - Family history
- Lifestyle
    - Diet
    - Physical activity
    - Healthy weight
    - Alcohol consumption
- Environmental factors
    - Radiation and chemical exposure
    - Socioeconomic status
- Most important for this thesis: Hormones
    - Estrogen
    - Timing of menarche and menopause, age at first birth
    - Hormone replacement therapy

\subsection{Diagnosis}

- Mammography
- Further imaging: Ultrasound, MRI
- Biopsy
- Stage is relevant

\subsection{Treatment}

- Depend on various factors
    - Stage
    - Subtype
    - Patient's health
- Surgery
    - Lumpectomy
    - Mastectomy
- Chemo- and radiotherapy
    - Reduce risk of recurrence, especially in more aggressive subtypes
- Hormone therapy
    - Tamoxifen and aromatase inhibitors
    - Effective in hormone receptor-positive subtypes
- Treatment choice is increasingly personalized
