\section{Breast cancer}
Breast cancer is a heterogeneous group of malignancies that primarily originates
in the breast tissue, characterized by the uncontrolled growth of cells. It is
the most prevalent cancer among women globally, accounting for approximately
25.2\% of all cancer cases in females and being the second leading cause of
cancer-related deaths among women\supercite{pace_breast_2016}. The disease can
manifest in various forms, with distinct biological behaviors and responses to
treatment. Notably, breast cancer can be classified based on the presence of
hormone receptors, such as estrogen and progesterone receptors, and the human
epidermal growth factor receptor 2 (HER2)\supercite{eccles_critical_2013}. The
understanding of these subtypes is crucial, as they significantly influence
prognosis and treatment strategies.

\subsection{Causes and risk factors}

Breast cancer is a complex disease, with many different causes and risk factors.
Understanding these factors is key to improving prevention, early detection, and
treatment. These risk factors can generally be grouped into genetic, lifestyle,
environmental, and — most relevant to this thesis — hormonal influences.

Genetics play a major role in breast cancer risk. Mutations in genes like BRCA1
and BRCA2 are well-known hereditary factors that greatly increase the chances of
developing breast cancer. Women with these mutations can face a lifetime risk of
up to 80\% for the disease\supercite{jian_clinical_2017}. Additionally, having a
family history of breast cancer significantly raises the risk, especially for
first-degree relatives of those affected\supercite{schairer_risk_2013}.

Lifestyle choices also impact breast cancer risk. Diets high in fat have been
linked to a higher risk, possibly because they affect estrogen
levels\supercite{turner_meta-analysis_2011}. On the other hand, regular exercise
and maintaining a healthy weight are protective
factors\supercite{claudia_admoun_etiology_2022}. Alcohol consumption is another
important factor, especially in estrogen receptor-positive breast
cancers\supercite{bao_association_2011}.

Environmental factors, such as exposure to radiation or certain chemicals, can
also raise the risk of breast cancer. For example, women who received radiation
therapy to the chest for other cancers have a significantly increased risk of
developing breast cancer later in life\supercite{froes_brandao_prolactin_2016}.
Additionally, socioeconomic status plays a role: women from lower-income
backgrounds often have less access to preventive care and early detection
services, which can lead to later diagnoses and worse
outcomes\supercite{cunningham_mind_2013}.

Hormonal influences are critical in breast cancer development. Factors such as
early menarche and late menopause increase risk because of longer exposure to
estrogen\supercite{nounu_sex_2022}. Reproductive history is another key factor:
women who have never given birth or had their first child later in life are at a
higher risk\supercite{claudia_admoun_etiology_2022}. Hormone replacement
therapy, particularly the combined use of estrogen and progestin, has also been
linked to an increased risk of breast cancer\supercite{turner_meta-analysis_2011}.

\subsection{Diagnosis}

Diagnosis of breast cancer typically involves a combination of clinical
evaluation, imaging studies, and histopathological examination. Initial
assessments often include mammography, which is a critical screening tool that
can detect tumors before they become palpable\supercite{hameed_breast_2020}. In
cases where abnormalities are found, further imaging, such as ultrasound or MRI,
may be employed. Definitive diagnosis is usually achieved through biopsy, where
tissue samples are collected and analyzed microscopically to confirm
malignancy\supercite{hameed_breast_2020}. The stage at which breast cancer is
diagnosed is a critical determinant of treatment outcomes; early-stage diagnosis
is associated with significantly better
prognoses\supercite{getachew_perceived_2020}. However, barriers to early
diagnosis, such as lack of awareness and access to healthcare facilities, can
lead to advanced-stage presentations, which are linked to poorer clinical
outcomes\supercite{getachew_perceived_2020,dickens_stage_2014}.

\subsection{Treatment}

Treatment options for breast cancer are multifaceted and depend on various
factors, including the cancer subtype, stage, and the patient's overall health.
Standard treatment modalities include surgery, radiation therapy, chemotherapy,
hormone therapy, and targeted therapies. Surgical options may range from
lumpectomy, which conserves breast tissue, to mastectomy, which involves the
removal of one or both
breasts\supercite{metcalfe_contralateral_2014,wu_breast_2014}. Adjuvant
therapies, such as chemotherapy and radiation, are often utilized to reduce the
risk of recurrence, especially in more aggressive forms like triple-negative
breast cancer, which is prevalent among women with BRCA1
mutations\supercite{metcalfe_contralateral_2014}. Hormonal therapies, such as
tamoxifen or aromatase inhibitors, are effective in hormone receptor-positive
breast cancers, while targeted therapies, including HER2 inhibitors, have
revolutionized treatment for HER2-positive
subtypes\supercite{eccles_critical_2013,pace_breast_2016}. The choice of
treatment is increasingly personalized, taking into account genetic and
epigenetic factors that may influence tumor behavior and patient response to
therapy\supercite{khakpour_methylomics_2017}.
