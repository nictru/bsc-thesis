\section{Breast cancer}
Breast cancer is a heterogeneous group of malignancies that primarily
originates in the breast tissue, characterized by the uncontrolled growth of
cells.
It is the most prevalent cancer among women globally, accounting for
approximately 25.2\% of all cancer cases in females and being the second
leading cause of cancer-related deaths among women\supercite{pace_breast_2016}.
The disease can manifest in various forms, with distinct biological behaviors
and responses to treatment.
Notably, breast cancer can be classified based on the presence of hormone
receptors, such as estrogen and progesterone receptors, and the human epidermal
growth factor receptor 2 (HER2)\supercite{eccles_critical_2013}.
The understanding of these subtypes is crucial, as they significantly influence
prognosis and treatment strategies.

\subsection{Types}

Breast cancer is a heterogeneous disease characterized by various subtypes that
differ in their molecular features, clinical behavior, and response to
treatment.
The classification of breast cancer is primarily based on the expression of
hormone receptors (estrogen and progesterone) and the human epidermal growth
factor receptor 2 (HER2).
Understanding these subtypes is crucial for determining prognosis and tailoring
treatment strategies.

One of the most common classifications is based on hormone receptor status,
which divides breast cancer into three main categories: hormone
receptor-positive (HR+), HER2-positive (HER2+), and triple-negative breast
cancer (TNBC).
HR+ breast cancers express either estrogen receptors (ER) or progesterone
receptors (PR) and are typically treated with hormone therapies such as
tamoxifen or aromatase inhibitors\supercite{geyer_molecular_2012}.
Within the HR+ category, there are further distinctions, such as luminal A and
luminal B subtypes.
Luminal A tumors are generally associated with a better prognosis and lower
proliferation rates, while luminal B tumors tend to be more aggressive and may
require chemotherapy in addition to hormone
therapy\supercite{geyer_molecular_2012}.

HER2-positive breast cancer is characterized by overexpression of the HER2
protein, which is associated with aggressive disease and poorer outcomes if
untreated.
However, the advent of targeted therapies, such as trastuzumab, has
significantly improved survival rates for patients with HER2+ breast
cancer\supercite{modi_antitumor_2020}.
Recent studies have also identified a subset of HER2-low breast cancer, which
may not meet the criteria for HER2 positivity but still expresses low levels of
the HER2 protein.
This subtype has been shown to have distinct clinical outcomes and may benefit
from specific therapeutic
approaches\supercite{won_clinical_2022,mutai_prognostic_2021}.

Triple-negative breast cancer (TNBC) is defined by the absence of ER, PR, and
HER2 expression.
This subtype is known for its aggressive nature and limited treatment options,
as it does not respond to hormone therapies or HER2-targeted
therapies\supercite{sizemore_triple_2021}.
TNBC can be further classified into several subtypes based on gene expression
profiles, including basal-like and claudin-low subtypes, which exhibit
different biological behaviors and responses to
treatment\supercite{lehmann_identification_2011}.
The lack of targeted therapies for TNBC has led to ongoing research into novel
treatment strategies, including immunotherapy and combination
therapies\supercite{lehmann_identification_2011}.

In addition to these major classifications, breast cancer can also be
categorized based on histological features, such as invasive ductal carcinoma
(IDC) and invasive lobular carcinoma (ILC).
IDC is the most common type, accounting for approximately 80\% of breast cancer
cases, while ILC is characterized by a distinct growth pattern and may present
unique challenges in diagnosis and
treatment\supercite{mittal_molecular_nodate}.

\subsection{Causes and risk factors}

Breast cancer is a complex disease, with many different causes and risk
factors.
Understanding these factors is key to improving prevention, early detection,
and treatment.
These risk factors can generally be grouped into genetic, lifestyle,
environmental, and — most relevant to this thesis — hormonal influences.

Genetics play a major role in breast cancer risk.
Mutations in genes like BRCA1 and BRCA2 are well-known hereditary factors that
greatly increase the chances of developing breast cancer.
Women with these mutations can face a lifetime risk of up to 80\% for the
disease\supercite{jian_clinical_2017}.
Additionally, having a family history of breast cancer significantly raises the
risk, especially for first-degree relatives of those
affected\supercite{schairer_risk_2013}.

Lifestyle choices also impact breast cancer risk.
Diets high in fat have been linked to a higher risk, possibly because they
affect estrogen levels\supercite{turner_meta-analysis_2011}.
On the other hand, regular exercise and maintaining a healthy weight are
protective factors\supercite{claudia_admoun_etiology_2022}.
Alcohol consumption is another important factor, especially in estrogen
receptor-positive breast cancers\supercite{bao_association_2011}.

Environmental factors, such as exposure to radiation or certain chemicals, can
also raise the risk of breast cancer.
For example, women who received radiation therapy to the chest for other
cancers have a significantly increased risk of developing breast cancer later
in life\supercite{froes_brandao_prolactin_2016}.
Additionally, socioeconomic status plays a role: women from lower-income
backgrounds often have less access to preventive care and early detection
services, which can lead to later diagnoses and worse
outcomes\supercite{cunningham_mind_2013}.

Hormonal influences are critical in breast cancer development.
Factors such as early menarche and late menopause increase risk because of
longer exposure to estrogen\supercite{nounu_sex_2022}.
Reproductive history is another key factor: women who have never given birth or
had their first child later in life are at a higher
risk\supercite{claudia_admoun_etiology_2022}.
Hormone replacement therapy, particularly the combined use of estrogen and
progestin, has also been linked to an increased risk of breast
cancer\supercite{turner_meta-analysis_2011}.

\subsection{Diagnosis}

Diagnosis of breast cancer typically involves a combination of clinical
evaluation, imaging studies, and histopathological examination.
Initial assessments often include mammography, which is a critical screening
tool that can detect tumors before they become
palpable\supercite{hameed_breast_2020}.
In cases where abnormalities are found, further imaging, such as ultrasound or
MRI, may be employed.
Definitive diagnosis is usually achieved through biopsy, where tissue samples
are collected and analyzed microscopically to confirm
malignancy\supercite{hameed_breast_2020}.
The stage at which breast cancer is diagnosed is a critical determinant of
treatment outcomes; early-stage diagnosis is associated with significantly
better prognoses\supercite{getachew_perceived_2020}.
However, barriers to early diagnosis, such as lack of awareness and access to
healthcare facilities, can lead to advanced-stage presentations, which are
linked to poorer clinical
outcomes\supercite{getachew_perceived_2020,dickens_stage_2014}.

\subsection{Treatment}

Treatment options for breast cancer are multifaceted and depend on various
factors, including the cancer subtype, stage, and the patient's overall health.
Standard treatment modalities include surgery, radiation therapy, chemotherapy,
hormone therapy, and targeted therapies.
Surgical options may range from lumpectomy, which conserves breast tissue, to
mastectomy, which involves the removal of one or both
breasts\supercite{metcalfe_contralateral_2014,wu_breast_2014}.
Adjuvant therapies, such as chemotherapy and radiation, are often utilized to
reduce the risk of recurrence, especially in more aggressive forms like
triple-negative breast cancer, which is prevalent among women with BRCA1
mutations\supercite{metcalfe_contralateral_2014}.
Hormonal therapies, such as tamoxifen or aromatase inhibitors, are effective in
hormone receptor-positive breast cancers, while targeted therapies, including
HER2 inhibitors, have revolutionized treatment for HER2-positive
subtypes\supercite{eccles_critical_2013,pace_breast_2016}.
The choice of treatment is increasingly personalized, taking into account
genetic and epigenetic factors that may influence tumor behavior and patient
response to therapy\supercite{khakpour_methylomics_2017}.
