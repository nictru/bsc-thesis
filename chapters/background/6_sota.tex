\section{State of the art}
Circular RNAs (circRNAs) have emerged as significant players in the molecular
landscape of breast cancer, particularly in relation to estrogen signaling and
tumor progression. These non-coding RNAs are generated through back-splicing and
have been implicated in various cellular processes, including the regulation of
gene expression and the modulation of signaling pathways associated with cancer
development\supercite{li_circrna-sfmbt2_2023,tran_new_2020}. The dysregulation
of circRNAs has
been observed in breast cancer, where they can act as competing endogenous RNAs
(ceRNAs), sponging microRNAs (miRNAs) and thereby influencing the expression of
target mRNAs involved in critical pathways such as estrogen receptor (ER)
signaling\supercite{nair_circular_2016,xu_circrna_2022}.

Recent studies have highlighted specific circRNAs that are associated with
breast cancer subtypes and their clinical implications. For instance,
circRNA-SFMBT2 has been shown to orchestrate ER\textalpha{} activation, contributing to
tamoxifen resistance in breast cancer cells, thereby underscoring the role of
circRNAs in therapeutic resistance\supercite{li_circrna-sfmbt2_2023}. Additionally, circRNAs such as
circTADA2As and circFBXL5 have been identified as regulators of miRNAs that
target key signaling pathways, including the SOCS3 and MAPK/ERK pathways, which
are crucial for cell proliferation and metastasis in breast
cancer\supercite{xu_circtada2as_2019,gao_hsa_circrna_0006528_2019}. This
suggests that circRNAs not only participate in the
pathogenesis of breast cancer but also serve as potential biomarkers for
diagnosis and prognosis\supercite{liu_influence_2021,chen_circepsti1_2018}.

The interaction between circRNAs and estrogen signaling is particularly
noteworthy. Research has demonstrated that ER-positive breast cancer subtypes
exhibit specific circRNA expression profiles that overlap with genes involved in
estrogen signaling pathways\supercite{nair_circular_2016}. For example, circRNA-000911 has
been shown to suppress breast cancer cell proliferation and invasion by sponging
miR-449a, which in turn activates the Notch1 signaling pathway, a known mediator
of estrogen signaling\supercite{wang_comprehensive_2018}. Furthermore, the circRNA-miRNA-mRNA
regulatory networks constructed in various studies have revealed that circRNAs
can modulate the expression of genes related to estrogen response, oxidative
stress, and epigenetic modifications, thereby influencing tumor behavior and
patient outcomes\supercite{xu_circrna_2022,nair_circular_2016}.
