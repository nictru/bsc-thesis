Breast cancer continues to be one of the most pressing challenges in modern
healthcare, representing the most frequently diagnosed cancer among women
worldwide.
In 2022 alone, approximately 2.3 million new cases were recorded, alongside
666,000 deaths, illustrating its substantial impact as a global health issue.
Accounting for 23.8\% of all cancer diagnoses in
women\supercite{bray_global_2024,ferlay_global_2024}, breast cancer remains the
leading cause of cancer-related death for women.
Projections suggest an alarming rise in incidence, with estimates reaching 4.4
million cases annually by 2070, placing even greater strain on healthcare
systems\supercite{lei_global_2021}.

Despite advancements in screening, early detection, and treatment, disparities
in access to care persist, particularly in low-resource settings.
These disparities contribute to late-stage diagnoses and higher mortality rates
in such regions\supercite{wilkinson_understanding_2022,ginsburg_breast_2020}.
Globally, the age-standardized mortality rate for breast cancer remains high at
approximately 12.65 deaths per 100,000 women per year, highlighting the ongoing
challenges in managing this complex
disease\supercite{bray_global_2024,ferlay_global_2024}.
Given the persistence of these challenges, significant efforts are required in
research, public health strategies, and healthcare delivery to improve patient
outcomes\supercite{desantis_breast_2019}.

At the molecular level, an emerging area of research involves circular RNAs
(circRNAs), a novel class of RNA molecules characterized by their covalently
closed loop structures.
Unlike linear RNAs, circRNAs exhibit increased stability and resistance to
degradation, making them reliable biomarkers and potential therapeutic targets
across a variety of diseases, including
cancer\supercite{ma_circular_2020,hoque_exploring_2023,wilusz_circular_2017}.
Initially considered byproducts of mRNA splicing, circRNAs are now recognized
for their regulatory roles in gene expression, cellular processes, and disease
mechanisms\supercite{cherubini_foxp1_2019,wilusz_360_2018}.

The unique characteristics of circRNAs—stability, abundance, and
tissue-specific expression—have made them attractive candidates for diagnostic
and therapeutic applications, particularly in cancer.
CircRNAs show promise as biomarkers for tumor progression and treatment
response\supercite{bao_prognostic_2020,ren_construction_2017}, and their
stability in bodily fluids enhances their utility in non-invasive liquid
biopsies\supercite{bao_prognostic_2020,zhang_circular_2018}.
Moreover, circRNAs such as hsa\_circ\_0000190 have been linked to advanced
stages of lung cancer, further demonstrating their potential in clinical
settings\supercite{luo_plasma_2020}.
Importantly, circRNAs have also been implicated in chemoresistance, providing
valuable insights into treatment efficacy and patient
management\supercite{geng_function_2018,feng_functions_2019}.

In breast cancer specifically, circRNAs have been identified as key players in
disease progression, particularly in relation to estrogen receptor (ER)
signaling.
Dysregulated circRNA expression has been observed in breast cancer, with
circRNAs functioning as competing endogenous RNAs (ceRNAs) that sponge miRNAs
involved in crucial pathways like ER
signaling\supercite{nair_circular_2016,xu_circrna_2022}.
Certain circRNAs, such as circRNA-SFMBT2, have been implicated in tamoxifen
resistance, a major hurdle in the treatment of ER-positive breast
cancer\supercite{li_circrna-sfmbt2_2023}.
Additionally, circRNAs like circTADA2As and circFBXL5 have been shown to
regulate miRNAs involved in key signaling pathways that promote cell
proliferation and metastasis, further highlighting their clinical
significance\supercite{xu_circtada2as_2019,gao_hsa_circrna_0006528_2019}.

Moreover, circRNA-miRNA-mRNA regulatory networks have been shown to modulate
gene expression related to estrogen response and other critical cellular
processes, influencing tumor behavior and patient outcomes in breast
cancer\supercite{xu_circrna_2022,nair_circular_2016}.
This makes circRNAs promising candidates for novel therapeutic approaches and
precision medicine in breast cancer treatment.

\section{Outline}

\begin{itemize}
      \item Describe role of breast cancer in modern health care
            \begin{itemize}
                  \item Abundance
                  \item Outcomes
            \end{itemize}
      \item Describe current methods for diagnosis and treatment
            \begin{itemize}
                  \item Gene panels for risk-assessment of already detected
                        tumors
                  \item Shortage of methods for detecting tumors on a molecular
                        basis
                        (need to do more research here)
            \end{itemize}
      \item Describe how circRNAs can improve diagnosis and treatment quality
            \begin{itemize}
                  \item Short description of circRNA structure
                  \item Increased stability compared to linear RNA biomarkers
                  \item Regulatory roles, potential drug targets
                  \item Help understanding how well existing treatment methods
                        work
                        (Letrozole, Tamoxifen)
            \end{itemize}
      \item Describe how this thesis approaches the investigation of circRNAs
            \begin{itemize}
                  \item Not sure how much detail is necessary here, and how
                        much
                        should
                        be left for materials and methods
            \end{itemize}
\end{itemize}