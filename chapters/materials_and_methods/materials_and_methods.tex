\section{Data}
The data used in this study were generated by P.
Furth et al.
in 2023
\supercite{furth_esr1_2023,furth_overexpression_2023}.

\subsection{Mouse models}
Mouse models allow the study of disease pathophysiology within the context of
physiological endocrine and immunologic function.

\subsection{Samples}
Describe the samples used in the study and how they were collected.
Include the timeline plots from the poster.
Explain the different sequencing steps.

\section{Nextflow and nf-core}
\paragraph{Nextflow} is a workflow management system that enables the
development of reproducible and scalable workflows.
It allows the creation of complex pipelines that can be executed on a variety
of platforms, from local machines to cloud computing environments.
Nextflow uses a domain-specific language (DSL) that simplifies the definition
of workflows and enables the reuse of existing components
\supercite{di_tommaso_nextflow_2017}.
As a result, Nextflow has become a popular tool in the bioinformatics
community.

\begin{figure}[ht]
    \centering

    \includegraphics[width=\textwidth]{chapters/materials_and_methods/figures/nextflow_usage.jpg}
    \caption{Workflow management systems} % TODO: Add detailed caption
    \label{fig:nextflow_usage}
\end{figure}

As pointed out by Langer et al.
in a recent preprint
\supercite{langer_empowering_2024}, programming-based workflow systems like
Nextflow and Snakemake have gained popularity during the last years, while
GUI-based systems like Galaxy have lost ground.
Furthermore, Nextflow has been the fastest growing workflow system in the last
years, with a remarkable 30 percent share of citations in 2023
(\cref{fig:nextflow_usage}).
The authors mostly attribute this to the great quality of the pipelines curated
by the nf-core community
\supercite{langer_empowering_2024,grayson_automatic_2023}.

\paragraph{nf-core} is a community-driven project that provides a collection of
high-quality, reproducible, and scalable Nextflow pipelines.
These pipelines cover a wide range of bioinformatics applications, from RNA-seq
and ChIP-seq to single-cell RNA-seq and metagenomics
\supercite{ewels_nf-core_2020}.
The community maintains a collection of reusable components, so that developers
can utilize them to speed up the development of new pipelines.
The nf-core project also provides guidelines for best practices in pipeline
development, ensuring that the resulting workflows are robust, efficient, and
easy to use \supercite{ewels_nf-core_2020}.

\section{nf-core/circrna}
The nf-core/circrna pipeline has originally been published by Digby et al.
in
2023 \supercite{digby_nf-corecircrna_2023}.
Since then, the pipeline has gone through several updates and improvements.
The pipeline can utilize seven different tools for BSJ detection, including
CIRIquant, CIRCexplorer2, circRNA finder, DCC, find\_circ, MapSplice, and
Segemehl.
It then annotates the detected circRNAs using GTF-based and database-based
annotation.
The pipeline also extracts the sequences of the circRNAs and quantifies their
expression levels together with the linear transcripts.
Finally, the pipeline performs miRNA interaction analysis using miRanda and
TargetScan, and provides several downstream analyses through a Shiny
application.
An overview of the pipeline is shown in \cref{fig:circrna_pipeline}.

\begin{figure}[ht]
    \centering

    \includegraphics[width=\textwidth]{chapters/materials_and_methods/figures/nf-core_circrna.png}
    \caption{nf-core/circrna} % TODO: Add detailed caption
    \label{fig:circrna_pipeline}
\end{figure}

\subsection{circRNA detection}
What is the main task here?

\subsubsection{CIRI2}
CIRI2 (CircRNA Identifier version 2) is an efficient algorithm designed for de
novo identification of circRNAs.
It utilizes a two-step approach: first, it identifies back-splice junction
reads, and then it filters these reads based on specific criteria to enhance
detection accuracy.
CIRI2 has been shown to outperform other tools in terms of sensitivity and
specificity, particularly in datasets with varying sequencing
depths\supercite{gao_ciri_2015,zheng_reconstruction_2019}.
This tool is particularly advantageous for large-scale circRNA studies due to
its unbiased nature and ability to handle complex
transcriptomes\supercite{chuang_assessing_2023}.

\subsubsection{CIRCexplorer2}
CIRCexplorer2 is another widely used tool that focuses on the detection of
circRNAs by analyzing RNA-seq data.
It employs a unique strategy that combines both back-splice junction reads and
linear RNA reads to improve circRNA identification.
CIRCexplorer2 has demonstrated robust performance in various studies, often
ranking high in comparative evaluations against other circRNA detection
tools\supercite{zeng_comprehensive_2017,nicolet_circular_2018}.
Its ability to provide detailed annotations and quantifications of circRNAs
makes it a valuable resource for researchers\supercite{hansen_comparison_2016}.

\subsubsection{circRNA finder}
The circRNA finder is a tool specifically designed for the identification of
circRNAs from RNA-seq data.
It employs a read mapping strategy that focuses on detecting chimeric reads,
which are indicative of circRNA formation.
This tool has been effectively utilized in various studies to identify circRNAs
across different biological contexts, although it may have limitations in
detecting circRNAs with non-canonical splice
signals\supercite{sekar_circular_2018,liu_prkra_2022}.
Its straightforward approach makes it accessible for researchers new to circRNA
analysis.

\subsubsection{DCC}
DCC (DCC: Detecting Circular RNAs) is a versatile software that allows for the
detection and quantification of circRNAs from RNA-seq data.
It computes expression levels of circRNAs independently of their linear
counterparts, which is crucial for understanding the functional roles of
circRNAs in various biological
processes\supercite{jakobi_profiling_2016,man_profiling_2020}.
DCC has been validated in numerous studies, demonstrating high reliability and
accuracy in circRNA detection\supercite{paraboschi_interpreting_2018}.
Its ability to integrate with other bioinformatics tools further enhances its
utility in comprehensive circRNA analyses.

\subsubsection{find\_circ}
Find\_circ is a widely recognized tool that identifies circRNAs by focusing on
back-splice junctions and employing a filtering strategy based on splice
signals.
While it has been effective in many studies, it may not capture circRNAs with
non-canonical splice signals, which can limit its detection
capabilities\supercite{sekar_circular_2018,liu_prkra_2022}.
Nonetheless, find\_circ remains a popular choice for researchers due to its
ease of use and integration with RNA-seq workflows.

\subsubsection{MapSplice}
MapSplice is primarily known for its role in splicing analysis but has also
been adapted for circRNA detection.
It utilizes a splice-aware alignment strategy to identify back-splice
junctions, making it suitable for circRNA studies.
However, its performance in circRNA detection may not be as robust as dedicated
circRNA tools like CIRI2 or
DCC\supercite{zeng_comprehensive_2017,chuang_nclscan_2016}.
MapSplice's strength lies in its ability to handle complex splicing events,
which can be beneficial in certain research contexts.

\subsubsection{Segemehl}
Segemehl is another tool that has been employed for circRNA detection,
particularly in studies involving complex transcriptomes.
It utilizes a unique alignment strategy that allows for the detection of both
linear and circular transcripts.
While Segemehl has shown promise in identifying circRNAs, its performance can
vary depending on the specific dataset and experimental
conditions\supercite{gao_ciri_2015,zeng_comprehensive_2017}.
Its flexibility in handling different types of RNA-seq data makes it a valuable
option for researchers exploring circRNA biology.

\subsection{circRNA annotation}
What is the main task here?

\subsubsection{GTF based annotation}
Describe GTF based annotation 

\subsubsection{Database based annotation} Describe database based annotation

\subsection{circRNA quantification} 

Why is this important?

\subsubsection{CIRIquant}
CIRIquant is an extension of the CIRI (CircRNA Identifier) framework, which
focuses on the accurate quantification of circRNAs by re-aligning back-splice
junction (BSJ) reads against a pseudo-reference sequence.
This approach significantly reduces the false discovery rate (FDR) associated
with circRNA identification, making it a robust choice for
researchers\supercite{zhang_accurate_2020}.
CIRIquant utilizes HISAT2 for aligning RNA-seq reads to the reference genome,
followed by the application of CIRI2 to identify putative circRNAs.
The integration of these methods allows for a comprehensive analysis of circRNA
expression, ensuring that only high-confidence circRNAs are
quantified\supercite{munz_exonintron_2021,made_circrna-mirna-mrna_2023}.
In studies, CIRIquant has been shown to correlate well with other
quantification methods, reinforcing its utility in circRNA
research\supercite{zhang_accurate_2020}.

\subsubsection{psirc-quant}
On the other hand, psirc-quant is a more recent tool that focuses on the
quantification of circRNAs derived from RNA-seq data.
It employs a unique strategy that combines circRNA identification with
expression quantification, allowing researchers to analyze circRNA dynamics in
various biological contexts.
Psirc-quant is particularly advantageous for its ability to handle large
datasets and provide insights into circRNA expression patterns across different
conditions\supercite{yu_quantifying_2021}.
This tool has been integrated into workflows that facilitate comprehensive
circRNA analyses, including differential expression studies and interaction
network construction\supercite{zhang_expression_2022}.

\subsection{miRNA interaction analysis}
The analysis of circular RNA (circRNA) interactions with microRNAs (miRNAs) is
crucial for understanding the regulatory roles of circRNAs in various
biological processes.
Two prominent tools used for predicting circRNA-miRNA interactions are MiRanda
and TargetScan.
These tools leverage sequence complementarity and binding energy calculations
to identify potential miRNA binding sites within circRNA sequences, thereby
elucidating their roles as competitive endogenous RNAs (ceRNAs).

\subsubsection{MiRanda}
MiRanda is a widely utilized algorithm that predicts miRNA targets based on
sequence complementarity and the stability of the RNA duplex formed between the
miRNA and its target.
It has been effectively employed in various studies to analyze circRNA-miRNA
interactions.
For instance, Vromman et al.
noted that
MiRanda is often used alongside TargetScan to predict miRNA binding sites in
circRNA sequences, contributing to a better understanding of circRNA functions
in gene regulation\supercite{vromman_closing_2021}.
Similarly, Zhang et al.
utilized
MiRanda in conjunction with TargetScan to predict microRNA response elements
(MREs) in differentially expressed circRNAs, demonstrating its utility in
identifying significant interactions\supercite{zhang_microarray_2017}.

\subsubsection{TargetScan}
TargetScan, on the other hand, focuses on the identification of conserved miRNA
binding sites across species, which enhances the reliability of the predicted
interactions.
This tool has been integrated into various studies to explore the regulatory
networks involving circRNAs and miRNAs.
For example, Jin et al.
employed MiRanda and TargetScan to predict interactions between circRNAs and
miRNAs, reinforcing the hypothesis that circRNAs can act as miRNA
sponges\supercite{jin_changes_2018}.
Furthermore, the combination of these tools allows researchers to construct
comprehensive circRNA-miRNA-mRNA networks, elucidating the complex regulatory
mechanisms at play in various
diseases\supercite{he_construction_2021,zhang_construction_2021}.

\subsection{Downstream analyses}
\subsubsection{R-shiny}
\subsubsection{Dimensionality reduction}
\subsubsection{Pathway analysis}
\subsubsection{Differential expression analysis}
\subsubsection{Genome browser}
